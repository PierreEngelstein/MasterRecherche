\documentclass[12pt,a4paper]{article}

\usepackage{mathtools}
\usepackage{graphicx}
\DeclarePairedDelimiter\bra{\langle}{\rvert}
\DeclarePairedDelimiter\ket{\lvert}{\rangle}
\DeclarePairedDelimiterX\braket[2]{\langle}{\rangle}{#1 \delimsize\vert #2}

\usepackage{amsthm}


\usepackage{lmodern} % police européennes  vectorielles  CM
\usepackage[utf8]{inputenc} % encodage à privilégier pour la  portabilité et +
\usepackage[frenchb]{babel} % francisation  de  libellés et de la typographie
\usepackage[T1]{fontenc} % encodage européen des caractères (Cork)8



\newtheorem{definition}{Définition}
\newtheorem{pb}{Problème}

\title{Algorithme de Deutsch-Jozsa}
\date{}


\begin{document}
\maketitle

\section{Problème à résoudre}
Soit une fonction $f$ définie par 
\[
  f: \{0, 1\}^n \to \{ 0, 1 \} \]
\[ 
(x_0, x_1, \dots , x_n) \mapsto y = f(x_0, x_1, \dots , x_n), 
\]

\begin{definition}
  Une fonction est dite équilibrée si $f$ retourne 0 pour la moitié de ses entrées.
\end{definition}

\begin{definition}
  Une fonction est dite constante si elle retourne 0 pour toutes ses
  entrées.
\end{definition}


\begin{pb}
Etant donnée une fonction $f$ qui est soit équilibrée, soit constante.
Le problème de Deutsch-Jozsa est de déterminer si $f$ est constante ou
non.  
\end{pb}

\subsection{Solution classique}
Dans le cas classique, il faut effectuer au pire $2^{n-1}+1$
évaluations pour déterminer si $f$ est constante ou équilibrée. Tout
d'abord, dès que deux évaluations sont différentes, $f$ est
nécessairement équilibrée. De plus, si après avoir évalué $2^{n-1}$
entrées et obtenu la même valeur, une évaluation supplémentaire nous
permet de connaitre dans quelle catégorie $f$ se trouve.

\subsection{Solution quantique}
Dans le cas quantique, ce problème se résout en une seule évaluation
quantique de $f$.

\begin{figure}[htbp]
    \centering
    \includegraphics[scale=0.2]{Deutsch-Jozsa_Algorithm.png}
    \caption{Schéma de l'algorithme}
    \label{fig:univerise}
\end{figure}

\subsubsection{Initialisation}
On commence avec :
$\ket{u_0} = \ket{0}^{\bigotimes n}\ket{1}$
: n-qubits à $\ket{0}$ et 1-qubit à $\ket{1}$

\subsubsection{Etape 1}

On applique une porte de Hadamard à $\ket{u_0}$ pour avoir un état équiprobable:
$\ket{u_1} = H\ket{u_0} = \frac{1}{\sqrt{2^{n + 1}}}
\displaystyle\sum_{x=0}^{2^n-1} \ket{x}(\ket{0} - \ket{1})$

\subsubsection{Etape 2}
On applique l'oracle quantique suivant à $\ket{u_1}$: $\ket{x}\ket{y}\rightarrow\ket{x}\ket{y\oplus f(x)}$

Prenons le cas à 1 qubit:
\[
f(x) = 0: \ket{x}(\ket{0} - \ket{1}) \rightarrow \ket{x}(\ket{0} - \ket{1})
\]
\[
f(x) = 1: \ket{x}(\ket{0} - \ket{1}) \rightarrow \ket{x}(\ket{1} - \ket{0})
\]
\[
f(x) quelconque: \ket{x}(\ket{0} - \ket{1}) \rightarrow (-1)^{f(x)}\ket{x}(\ket{0} - \ket{1})
\]

En généralisant:

$\ket{u_2} = \frac{1}{\sqrt{2^{n + 1}}}
\displaystyle\sum_{x=0}^{2^n-1} (-1)^{f(x)}\ket{x}(\ket{0} - \ket{1})$

On peut ignorer le dernier qubit ($\ket{0} - \ket{1}$) comme il est
constant. Finalement, on en déduit :

$\ket{u_2} = \frac{1}{\sqrt{2^{n + 1}}}
\displaystyle\sum_{x=0}^{2^n-1} (-1)^{f(x)}\ket{x}$

\subsubsection{Etape 3}
On réapplique une porte Hadamard à chaque qubit sortant, ce qui donne:

$\ket{u_3} = \frac{1}{\sqrt{2^{n}}}
\displaystyle\sum_{x=0}^{2^n-1} (-1)^{f(x)} \displaystyle\sum_{y=0}^{2^n-1} (-1)^{x.y}\ket{y}$

$\ket{u_3} = \frac{1}{\sqrt{2^{n}}}
\displaystyle\sum_{x=0}^{2^n-1} \displaystyle\sum_{y=0}^{2^n-1}(-1)^{f(x)} (-1)^{x.y}\ket{y}$

La probabilité de mesurer $\ket{0}^{\bigotimes n}$ est : 
$|\frac{1}{\sqrt{2^{n}}}\displaystyle\sum_{x=0}^{2^n-1}(-1)^{f(x)}|$

Si on obtient 0, alors $f(x)$ est constante. Si on obtient 1, alors $f(x)$ est équilibrée.

\end{document}
%%% Local Variables:
%%% mode: latex
%%% TeX-master: t
%%% End:
