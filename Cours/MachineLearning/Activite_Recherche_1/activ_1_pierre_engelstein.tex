\documentclass[12pt,a4paper]{article}

\usepackage{mathtools}
\usepackage{graphicx}
\usepackage{amsthm}
\usepackage{lmodern} % police européennes  vectorielles  CM
\usepackage[utf8]{inputenc} % encodage à privilégier pour la  portabilité et +
\usepackage[frenchb]{babel} % francisation  de  libellés et de la typographie
\usepackage[T1]{fontenc} % encodage européen des caractères (Cork)8


\title{Activité de recherche 1: Lecture d'article académique}
\author{Pierre Engelstein}
\date{}


\begin{document}
\maketitle

Article choisi: Multicomponent and Longitudinal Imaging Seen as a Communication Channel—An Application to Stroke

\section{Type d'article}

Cet article a été publié dans le journal \textit{Entropy} en Avril 2017. C'est un article original, disponible en Open-Access grâce au publicateur MDPR (Multidisciplinary Digital Publishing Institute). Ce journal utilise l'évaluation par les pairs, spécifiquement de la revue à l'aveugle, où les auteurs ne connaissent pas les noms des évaluateurs.

\section{Problème considéré}

Ce papier nous présente la problématique de pouvoir prédire l'évolution de santé d'un tissu organique (comme le cerveau) à l'aide d'images médicales prises au préalable, dans le but de prédire et d'éviter de futurs accidents. Il applique spécifiquement ce problème aux AVC (Accidents Vasculaires Cérébraux) où des signes préalables peuvent être visibles sur les patients; et se pose la question des problématiques de bruit dans les mesures empêchant potentiellement de bons diagnostics.

\section{En quoi le problème est-il important ?}

\section{Solution proposée}

\section{Originalité de la solution}

\section{Hypothèses, limitations, conditions et cadre nécessaires pour application de la solution}

\end{document}