\documentclass[12pt,a4paper]{article}
\usepackage{mathtools}
\usepackage{graphicx}
\usepackage{amsthm}
\usepackage{lmodern} % police européennes  vectorielles  CM
\usepackage[utf8]{inputenc} % encodage à privilégier pour la  portabilité et +
\usepackage[frenchb]{babel} % francisation  de  libellés et de la typographie
\usepackage[T1]{fontenc} % encodage européen des caractères (Cork)8

\title{Activité de recherche 1: Lecture d'article académique}
\author{Pierre Engelstein}
\date{}
\begin{document}
\maketitle

Article choisi: Multicomponent and Longitudinal Imaging Seen as a Communication Channel—An Application to Stroke

\section{Type d'article}

Cet article a été publié dans le journal \textit{Entropy} en Avril 2017. C'est un article original, disponible en Open-Access grâce au publicateur MDPR (Multidisciplinary Digital Publishing Institute). Ce journal utilise l'évaluation par les pairs, spécifiquement de la revue à l'aveugle, où les auteurs ne connaissent pas les noms des évaluateurs.

\section{Problème considéré}

Ce papier nous présente la problématique de pouvoir prédire l'évolution de santé d'un tissu organique (comme le cerveau) à l'aide d'images médicales prises au préalable, dans le but de prédire et d'éviter de futurs accidents. Il applique spécifiquement ce problème aux AVC (Accidents Vasculaires Cérébraux) où des signes préalables peuvent être visibles sur les patients; et se pose la question des problématiques de bruit dans les mesures empêchant potentiellement de bons diagnostics.

\section{En quoi le problème est-il important ?}
La détection et le traitement d'AVC est très problématique. Tout d'abord, les causes précises biologiques de ce phénomène sont toujours mal comprises, et ensuite les traitements proposés peuvent avoir des effets secondaires non désirés. Afin d'améliorer les diagnostics, et les traitements, l'utilisation de l'imagerie médicale peut voir un vrai impact positif.


\section{Solution proposée}

Ce papier propose d'étudier l'évolution de la santé des tissus organiques en prenant l'angle de vue d'un canal de communication de Shannon. Il se base sur l'utilisation de la théorie de l'information pour mesurer l'évolution temporelle d'un processus donnée. Spécifiquement, il propose d'utiliser les données suivantes:

\begin{itemize}
    \item L'entrée $X$ constituée d'une pile d'images 3D prises à différents stades de la maladie. Ces images sont composées de plusieurs composantes telles que l'image prise par IRM, ou le flux sanguin cérébral, le volume de sang dans le cerveau, etc.
    \item La sortie $Y$ après l'ensemble des traitements, représentant une estimation de l'infarctus final.
\end{itemize}

En revanche, les auteurs précisent que cette entrée et cette sortie sont soumises à beaucoup de bruit du fait des conditions de capture: le patient, soumis à une intense douleur ou un choc, a tendance à bouger lors de la capture des images, entraînant une représentation 3D non parfaite.

\medbreak

Les auteurs présentent ici trois éléments à leur méthodologie:

\begin{itemize}
    \item La mesure du gain de prédictabilité. En utilisant la régression statistique pas à pas, et en simulant du bruit sur les entrées $X_i$
    \item La prise en compte de l'environnement de chaque voxel. Au lieu de considérer chaque voxel (pixel en 3 dimensions) individuellement, les auteurs proposent d'encoder dans les voxels de l'information mesurant leur environnement. Entre autres, ils proposent de d'encoder la présence où non de voxels voisins dans une zone affectée par l'AVC.
    \item La mesure de l'impact du bruit sur la précision des prédictions de l'avancement de la santé des tissus. Cela représente une difficulté car on a pas accès aux mesures non bruitées. Pour résoudre ce problème, les auteurs proposent de simuler le cas où l'entrée $X$ permet de décrire entièrement $Y$ (quand $X$ n'est pas bruité), d'ajouter progressivement du bruit à une entrée $X$, et de mesurer les conséquences sur $Y$ par la suite.
\end{itemize}

\section{Originalité de la solution}

Cette solution présente la mise en place d'un bruit sur la segmentation des images réaliste, ce qui n'a, a priori, pas été présenté auparavant. Ici, les auteurs indiquent qu'au lieu de mettre de l'aléatoire dans les captures, on peu effectuer de l'aléatoire local. En principe, cela consiste à créer des masques binaires appliqués à l'image, qui vont soit enlever soit ajouter de la donnée aux voxels. De plus, les auteurs ajoutent à cette méthode que la modification par le masque d'un binaire va entrainer un changement de probabilité des voxels voisins d'être modifiés par le masque, dans le but de garder des bruits locaux.

\section{Hypothèses, limitations, conditions et cadre nécessaires pour application de la solution}

Cette étude utilise une base de données de données médicales, dont les patients ont donné l'accord. Au total, les données correspondent à 9 patients.

Une des limitations de cette étude est la taille des données. En effet, afin d'obtenir des résultats corrects, la taille des données doit être conséquente. Cela demande des puissances de calculs pour le traitement des données, ce qui n'est pas facilement accessible en pratique.

\end{document}