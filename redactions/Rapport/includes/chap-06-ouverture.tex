\chapter{Ouverture vers le stage}

A la suite de ce travail, un certain nombre de questions restent ouvertes et serviront de pistes de travail pour le stage qui suit:

\begin{itemize}
    \item Sur la construction effective des circuits, est-il optimal de décomposer le circuit en portes élémentaires CNOT ou est-il plus rapide de juste exécuter le problème classiquement ? La question se pose en particulier pour l'algorithme de Deutsch-Jozsa où l'oracle a besoin, a priori, d'être construit en connaissant déjà toutes les sorties de la fonction.
    \item Sur l'algorithme de Deutsch-Jozsa encore une fois, que se passe-t-il quand on l'applique à d'autres classes de fonction, et comment peut-on l'adapter pour différencier d'autres types de fonctions ?
    \item Sur le problème de Deutsch-Jozsa, peut-on trouver un autre algorithme permettant de résoudre le problème ? Cela amène vers l'écriture des contraintes algébriques necessaires à la résolution du problème, et d'essayer de déterminer si la solution proposée par Deutsch-Jozsa est optimale ou non.
    \item En étendant le point précédent, peut-on poser des contraintes algébriques sur les problèmes ou écrire des spécifications algébriques pour les différents algorithmes ? Cela permettrait de caractériser systématiquement les problèmes pouvant disposer d'une amélioration quantique.
\end{itemize}