\chapter{Informatique quantique: éléments de base}

\section{Le qubit}

\subsection{Représentation d'état}
Prenons un système composé 

\subsection{Formalisme mathématique}

\subsubsection*{Qubit unique}

Le qubit possède deux états de base, correspondant aux états des bits classiques. On les représente par $\ket{0}$, correspondant à l'état $0$ classique, et par $\ket{1}$ pour l'état $1$ classique. A la différence d'un bit classique, un qubit peut également prendre une infinité d'autres états que ses états de base. La question se pose alors de la mesure: que va-t-on mesurer quand un qubit est dans un état autre que $\ket{0}$ ou $\ket{1}$ ? C'est là qu'apparaissent les bizarreries de la mécanique quantique. La mesure va donner au hasard $0$ ou $1$, suivant des probabilités définies.

Pour représenter ce comportement, on note un qubit de la façon suivante:

\begin{equation}
\ket{\psi} = \alpha \cdot \ket{0} + \beta \cdot \ket{1}
\end{equation}

Le qubit est alors représenté par une combinaison linéaire des deux états de base $\ket{0}$ et $\ket{1}$, suivant les coefficients complexes $\alpha$ et $\beta$. Ces coefficients représentent les amplitudes de probabilité suivant lesquelles on va mesurer $\ket{0}$ ou $\ket{1}$.

Ces deux coefficients complexes doivent absolument respecter la propriété suivante:

\begin{equation}
\norm{\alpha}^2 + \norm{\alpha}^2 = 1
\end{equation}    

\subsubsection*{Multiples qubits}

Une fois ces éléments posés, on peut commencer à travailler avec plusieurs qubits, notés n-qubits. Mathématiquement, une combinaison de qubits correspond à un produit tensoriel de deux vecteurs.

Prenons les qubits $\ket{0}$ et $\ket{1}$. La combinaisons en un 2-qbits est donc:

\begin{equation}
    \ket{\psi} = \ket{0} \tens{} \ket{1}
\end{equation} 

qu'on peut écrire plus simplement:

\begin{equation}
    \ket{\psi} = \ket{01}
\end{equation}

Un 2-qubit a donc 4 états de bases, représentés par: $\{\ket{00},\ket{01}, \ket{10}, \ket{11} \}$, et peut donc être la combinaison linéaire de n'importe quel de ces états de base

\subsubsection*{Intrication}
Prenons un 2-qubit formé par la combinaison de 2 qubits:

\begin{align*}
\ket{\psi} &= (\alpha_1\cdot\ket{0} + \beta_1\cdot\ket{1}) \otimes (\alpha_2\cdot\ket{0} + \beta_2\cdot\ket{1}) \\
&= \alpha_1\alpha_2 \ket{0} \otimes \ket{0} + \alpha_1\beta_2 \ket{0} \otimes \ket{1} + \beta_1\alpha_2 \ket{1} \otimes \ket{0} + \beta_1\beta_2 \ket{1} \otimes \ket{1} \\
&= \gamma_1 \ket{00} + \gamma_2 \ket{01} + \gamma_3 \ket{10} + \gamma_4 \ket{11}
\end{align*}

On peut donc, si on a un 2-qubit combinaison linéaire de tout les états de bases, le séparer en deux qubits individuels, sur lesquels on va pouvoir agir.

\medbreak

Considérons maintenant le 2-qubit suivant:

\[
\ket{\psi} = \gamma_1 \ket{00} + \gamma_2 \ket{11}
\]

Il parait évident alors qu'on ne peut pas séparer ce 2-qubit en produit tensoriel de 2 qubits individuels. Dans ce cas, on dit que les deux qubits sont \textbf{intriqués} et donc non séparables.

\section{Représentation géométrique}

\section{Programmer un processeur quantique}