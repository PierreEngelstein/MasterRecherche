\chapter{Introduction}

L'informatique quantique, application des théories quantiques développées depuis le début du vingtième siècle à la théorie de l'information puis spécifiquement au calcul, est aujourd'hui en plein développement, théorique avec la découverte de nouveaux algorithmes, mais aussi pratique avec l'intérêt porté par les différents industriels. On voit alors l'apparition de nouvelles plateformes basées sur des processeurs quantiques, permettant de mettre en place les différentes avancées théoriques.

L'informatique quantique est en effet d'intérêt car cela permet une accélération de certains traitements permettant théoriquement d'effectuer des calculs qui seraient infaisable en des temps raisonnables sur nos calculateurs classiques.

Avec ce nouveau champ, de nombreuses questions se posent sur nos infrastructures actuelles, notamment en termes de cybersécurité et il est alors important de comprendre les capacités qu'offre l'informatique quantique.

Ce travail présente donc tout d'abord les notions fondamentales à la compréhension de l'informatique quantique. On y montre tout d'abord ce qu'est un qubit, puis on explique le mécanisme de mesure qui vient apporter de la probabilité; et enfin les principes d'évolution de systèmes quantiques permettant de construire des systèmes de calcul.

La deuxième partie de ce travail présente un tableau des mises en oeuvres expérimentales au travers des processeurs quantiques et des simulateurs, développés par les différents industriels depuis le début des années 2010.

Enfin, la dernière partie présente trois algorithmes majeurs au développement de l'informatique quantique. On explique tout d'abord l'algorithme de Deutsch-Jozsa, sur la parallélisation d'évaluation de fonction; puis l'algorithme de Grover, sur la recherche de base de données; et enfin l'algorithme de Shor sur la factorisation de nombres premiers.