\chapter{Algorithmes dérivés de Deutsch-Jozsa}
\section{Problème à résoudre}

Soient $x$ et $s$ tels que $x, s \in \{0, 1\}^n$.

On pose une fonction $f$ définie par:
\[
  \begin{array}{llll}
    f :  &  x              & \to     & y = s \cdot x \Mod{2} = x_1 s_1 + x_2 s_2 + \dots + x_n s_n \\
    f:   &  \{0, 1\}^n     & \to     & \{0, 1\},
  \end{array}  
\]

\begin{ex}
  Soit $s$ le mot booléen suivant: $s = 10$. La fonction $f$ a donc la table de véritée suivante:
\[
  \begin{array}{|c|c|c|}
    \hline
   (x_1, x_2) & s & f(x_1,x_2) \\
    \hline
    (0,0) & 10 & 0 \\
    \hline
    (0,1) & 10 & 0 \\
    \hline
    (1,0) & 10 & 1 \\
    \hline
    (1,1) & 10 & 1 \\
    \hline
  \end{array}
\]
On observe que le résultat est de 1 pour les entrées $(x_1, x_2)$ où l'emplacement des $1$ correspond à ceux de $s$.
\end{ex}

\begin{pb}[Bernstein-Vazirani]
Etant donné un mot $s$ secret, et la fonction $f$ implémentant l'opération décrite précédemment, comment peut on retrouver $s$ en le moins d'évaluations de $f$ possibles ?
\end{pb}

\subsection{Solution classique}
Dans le cas classique, on va devoir évaluer au pire toutes les valeurs possibles de $s$ pour trouver sa valeur, soit $n$ évaluations de $f$. C'est un algorithme de complexité $\mathcal{O}(n)$

\subsection{Solution quantique}
Dans le cas quantique, ce problème se résout en une seule évaluation
quantique de $f$. L'algorithme reprends celui de Deutsch-Jozsa en changeant la fonction appliquée dans l'oracle quantique.

\subsubsection{Initialisation}
On commence avec :
$\ket{u_0} = (\ket{0}^{\bigotimes n})$
: n-qubits à $\ket{0}$

\subsubsection{Etape 1}

On applique une porte de Hadamard à $\ket{u_0}$ pour avoir un état équiprobable:
$\ket{u_1} = H\ket{u_0} = \frac{1}{\sqrt{2^n}}
\displaystyle\sum_{x=0}^{2^n-1} \ket{x}$

\subsubsection{Etape 2}
On applique l'oracle quantique suivant à $\ket{u_1}$:
\[ o : \ket{x}\ket{y}\mapsto \ket{x}\ket{y\oplus (s \cdot x \Mod{2})}. \]

En suivant exactement le même raisonnement que pour Deutsch-Jozsa, on arrive à l'expression suivante: 

\begin{equation}\ket{u_2} = \frac{1}{\sqrt{2^n}}
\displaystyle\sum_{x=0}^{2^n-1} (-1)^{s \cdot x \Mod{2}}\ket{x} 
\end{equation}

\subsubsection{Etape 3}

De la même façon à Deutsch-Jozsa, on applique une porte Hadamard à chaque qubit sortant, ce qui donne:

\[ \ket{u_3} = \frac{1}{\sqrt{2^{n}}}
\displaystyle\sum_{x=0}^{2^n-1} (-1)^{s \cdot x \Mod{2}} \left( \frac{1}{\sqrt{2^{n}}} \displaystyle\sum_{y=0}^{2^n-1} (-1)^{x.y}\ket{y} \right) \]

\begin{align}
  \ket{u_3} &= \frac{1}{\sqrt{2^{n}}}\displaystyle\sum_{x=0}^{2^n-1} (-1)^{s \cdot x \Mod{2}} \left( \frac{1}{\sqrt{2^{n}}} \displaystyle\sum_{y=0}^{2^n-1} (-1)^{x.y}\ket{y} \right) \\
  & = \frac{1}{2^{n}} \displaystyle\sum_{x=0}^{2^n-1} \displaystyle\sum_{y=0}^{2^n-1} (-1)^{(s \cdot x \Mod{2}) + x.y} \ket{y}
  \end{align}

Et on peut prouver que $\frac{1}{2^{n}} \displaystyle\sum_{x=0}^{2^n-1} \displaystyle\sum_{y=0}^{2^n-1} (-1)^{(s \cdot x \Mod{2}) + x.y} \ket{y}$ est égal à $\ket{s}$ (\textit{à faire ...})

\subsection{Exemple}

Prenons par exemple $s=(10)_2 = 2_{10}$, soit $f(x) = 2 \cdot x \Mod{2}$
\subsubsection*{Etape 1: porte de Hadamard}

On commence avec $\ket{u_0} = \ket{00}$. La première étape est
l'application de la porte d'hadamard à $\ket{u_0}$:

\begin{align}
\ket{u_1} &= H\ket{u_0} = H\ket{0} \otimes H\ket{0} \\
& = \frac{1}{2} \left( (\ket{0} + \ket{1})\otimes(\ket{0} + \ket{1}) \right) \\
 & = \frac{1}{2}\{\ket{00} + \ket{01} + \ket{10} + \ket{11}\}
\end{align}

%On peut factoriser le tout par $(\ket{0} - \ket{1})$: 
%$

\subsubsection*{Etape 2: oracle quantique}

On applique à $\ket{u_1}$ l'oracle quantique $\ket{x}\ket{y}\rightarrow\ket{x}\ket{y\oplus (s \cdot x \Mod{2})} = :$

\begin{align*}
  \ket{u_2} &= \frac{1}{2}((-1)^{10 \cdot 00 \Mod{2}} \ket{00} + (-1)^{10 \cdot 01 \Mod{2}} \ket{01} + (-1)^{10 \cdot 10 \Mod{2}} \ket{10} + (-1)^{10 \cdot 11 \Mod{2}} \ket{11}) \\
  &= \frac{1}{2}((-1)^{0} \ket{00} + (-1)^{0} \ket{01} + (-1)^{1} \ket{10} + (-1)^{1} \ket{11}) \\
  &= \frac{1}{2} (\ket{00} + \ket{01} - \ket{10} - \ket{11})
\end{align*}

\subsubsection*{Etape 3: porte de Hadamard}

On applique donc une porte de hadamard à $\ket{u_2}$:
\begin{equation}
  \label{eq:5}
\ket{u_3} = \frac{1}{2} H \left( (\ket{00} + \ket{01} - \ket{10} + \ket{11}) \right) 
\end{equation}

Nous sommes sur une porte de hadamard pour 2 qubits, ce qui donne
la relation matricielle suivante pour $\ket{u_3}$:

\begin{align}
\ket{u_3} &=
\frac{1}{4} 
\begin{bmatrix}
  1 & 1 & 1 & 1 \\
  1 & -1 & 1 & -1 \\
  1 & 1 & -1 & -1 \\
  1 & -1 & -1 & 1 \\
\end{bmatrix}
\begin{bmatrix}
  1 \\ 1 \\ -1 \\ -1
\end{bmatrix} , \\ 
 &= \frac{1}{4} 
\begin{bmatrix}
  0 \\
  0 \\
  4 \\
  0 \\
\end{bmatrix}.
\end{align}

Lors de la mesure, on va obtenir l'état $\ket{10}$ avec une probabilité de 1, qui était bien notre mot binaire $s$ de départ.

On peut observer que, lors de l'application de la porte de Hadamard à $\ket{u_2}$, on obtient la superposition d'état suivante: $\ket{00} + \ket{01} - \ket{10} - \ket{11}$. Cela correspond à la troisième ligne de la matrice de Hadamard, correspondant au $\ket{s}$ voulu. Dans tout les cas, peut importe le $s$ choisi, on obtiendra une superposition d'état correspondant à une des lignes de la matrice, forçant à 0 les probabilités de tout les états sauf de celui indiqué.

\subsection{Implémentation du circuit}

\subsubsection*{Circuit global}
L'implémentation du circuit quantique pour cet algorithme est très similaire à celui de Deutsch-Jozsa, à la différence qu'on a un qubit de moins:

\centerline{
  \Qcircuit @C=1em @R=.7em {
    & \gate{H} & \multigate{2}{U_f} & \gate{H} & \meter & \qw \\
    & \gate{H} & \ghost{U_f}& \gate{H} & \meter & \qw \\
    & \gate{H} & \ghost{U_f} & \gate{H} & \meter & \qw
  }
}
\subsubsection*{Implémentation de l'oracle}

Prenons le cas où $n=2$. La matrice correspondant à la porte $U_f$ va avoir 4 possibilité pour obtenir, comme on l'a dit lors de l'exemple, une des 4 lignes de la matrice de Hadamard:
\[
U_{f_{00}} = 
\begin{bmatrix}
  1 & 0 & 0 & 0 \\
  0 & 1 & 0 & 0 \\
  0 & 0 & 1 & 0 \\
  0 & 0 & 0 & 1 \\
\end{bmatrix}
, U_{f_{01}} = 
\begin{bmatrix}
  1 & 0 & 0 & 0 \\
  0 & -1 & 0 & 0 \\
  0 & 0 & 1 & 0 \\
  0 & 0 & 0 & -1 \\
\end{bmatrix}
,U_{f_{10}} = 
\begin{bmatrix}
  1 & 0 & 0 & 0 \\
  0 & 1 & 0 & 0 \\
  0 & 0 & -1 & 0 \\
  0 & 0 & 0 & -1 \\
\end{bmatrix}
, U_{f_{11}} = 
\begin{bmatrix}
  1 & 0 & 0 & 0 \\
  0 & -1 & 0 & 0 \\
  0 & 0 & -1 & 0 \\
  0 & 0 & 0 & 1 \\
\end{bmatrix}
\]

On remarque que ces quatres matrices sont en fait des produits tensoriels de deux matrices correspondant à des portes à 1 qubit:

\[
  I = 
  \begin{bmatrix}
    1 & 0 \\
    0 & 1 \\
  \end{bmatrix}
  , Z = 
  \begin{bmatrix}
    1 & 0 \\
    0 & -1 \\
  \end{bmatrix}
\]

Pour $n=2$, on a $s \in \{00, 01, 10, 11\}$. En reprenant les matrices correspondantes, on obtient les produits tensoriels suivant:

\[
U_{f_{00}} = I \otimes I
, U_{f_{01}} = I \otimes Z
,U_{f_{10}} = Z \otimes I
, U_{f_{11}} = Z \otimes Z
\]

On peut généraliser sur l'implémentation en disant:

\begin{align}
  U_f = \displaystyle \bigotimes_{i=0}^{n} U_i, \; U_i = \begin{cases}
    I  & \text{si } s_i = 0 \\
    Z  & \text{si } s_i = 1 \\
  \end{cases}
\end{align}

Un exemple d'implémentation complète serait alors (pour $s = 101$):

\centerline{
  \Qcircuit @C=1em @R=.7em {
    & \lstick{\ket{0}} & \gate{H} \barrier[-1.25em]{2} & \gate{Z} \barrier[-1.25em]{2} & \gate{H} & \meter & \qw \\
    & \lstick{\ket{0}} & \gate{H} & \qw & \gate{H} & \meter & \qw \\
    & \lstick{\ket{0}} & \gate{H} & \gate{Z} & \gate{H} & \meter & \qw
  }
}