\chapter{Conclusion}

Ce travail bibliographique met ainsi en place les différents éléments necessaires à la compréhension de l'informatique quantique: la notion de système quantique, amenant la définition de qubit, les mécanismes permettant de faire évoluer les systèmes, ainsi que la mesure. On illustre de plus trois algorithmes majeurs dans l'histoire de ce domaine, en montrant leur champs d'application, les problèmes qu'ils permettent de résoudre. De tout ce travail découle la conclusion que l'informatique quantique permet d'accélérer considérablement la résolution d'un certain nombre de problèmes impossibles à réoudre avec les technologies classiques.

En revanche, un certain nombre de problèmes restent ouverts à la suite de ce travail:

\begin{itemize}
    \item Sur la construction effective des circuits, est-il optimal de décomposer le circuit en portes élémentaires CNOT ou est-il plus rapide de juste exécuter le problème classiquement ? La question se pose en particulier pour l'algorithme de Deutsch-Jozsa où l'oracle a besoin, a priori, d'être construit en connaissant déjà toutes les sorties de la fonction.
    \item Sur l'algorithme de Deutsch-Jozsa encore une fois, que se passe-t-il quand on l'applique à d'autres classes de fonction, et comment peut-on l'adapter pour différencier d'autres types de fonctions ?
    \item Sur le problème de Deutsch-Jozsa, peut-on trouver un autre algorithme permettant de résoudre le problème ? Cela amène vers l'écriture des contraintes algébriques necessaires à la résolution du problème, et d'essayer de déterminer si la solution proposée par Deutsch-Jozsa est optimale ou non.
    \item En étendant le point précédent, peut-on poser des contraintes algébriques sur les problèmes ou écrire des spécifications algébriques pour les différents algorithmes ? Cela permettrait de caractériser systématiquement les problèmes pouvant disposer d'une amélioration quantique.
\end{itemize}