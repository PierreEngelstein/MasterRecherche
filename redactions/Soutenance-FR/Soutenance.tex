\documentclass{beamer}
\usetheme{Frankfurt}
\addtobeamertemplate{navigation symbols}{}{%
    \usebeamerfont{footline}%
    \usebeamercolor[fg]{footline}%
    \hspace{1em}%
    \insertframenumber/\inserttotalframenumber
}

\usepackage{mathtools}
\usepackage{graphicx}
\usepackage{braket}
\usepackage{amsthm}
\usepackage{lmodern}
\usepackage[utf8]{inputenc}
\usepackage[frenchb]{babel}
\usepackage[T1]{fontenc}
\usepackage{subcaption}
\usepackage{caption}
\usepackage{gensymb}
\usepackage{tikz}
\usepackage[qm]{qcircuit}
\usepackage{listings}
\usepackage{pgfplots}
\usepackage{xcolor}
\usepackage[ruled,vlined]{algorithm2e}

\definecolor{mGreen}{rgb}{0,0.6,0}
\definecolor{mGray}{rgb}{0.5,0.5,0.5}
\definecolor{mPurple}{rgb}{0.58,0,0.82}
\definecolor{backgroundColour}{rgb}{0.95,0.95,0.92}

\lstdefinestyle{CStyle}{
    backgroundcolor=\color{backgroundColour},   
    commentstyle=\color{mGreen},
    keywordstyle=\color{magenta},
    numberstyle=\tiny\color{mGray},
    stringstyle=\color{mPurple},
    basicstyle=\footnotesize,
    breakatwhitespace=false,         
    breaklines=true,                 
    captionpos=b,                    
    keepspaces=true,                 
    numbers=left,                    
    numbersep=5pt,                  
    showspaces=false,                
    showstringspaces=false,
    showtabs=false,                  
    tabsize=2,
    language=Python
}


\newtheorem{pb}{Problème}
\newtheorem{rem}{Remarque}
\newtheorem{ex}{Exemple}
% \newtheorem{definition}{Définition}

\titlegraphic{}

\title{Soutenance de rapport bibliographique - Informatique quantique}
\author{Pierre Engelstein}
% Rajouter nom des encadrants + membres du jury
\institute{Polytech Angers}

\begin{document}

%====================================
%================1===================

\frame{\titlepage}

%% Intro

% Notion de Qubit
% Combinaison de qubits: le produit tensoriel
% Mesure des qubits
% Représentations graphiques

%% Partie 1: Algorithme de Deutsch Jozsa

% Présentation du problème
% Notion d'oracle
% Explication algébrique
  % Porte de hadamard: se porter dans un état superposé
  % Application de l'oracle
  % Réapplication de Hadamard: forcer les résultats
% Implémentation en QSharp: expliquer un exemple d'oracle
% Visualisation géométrique

%% Partie 2: Algorithme de Grover

% Présentation du problème
% Miroir géométrique
% Application d'une itération
  % Opérateur s et w
  % Obtention d'un état probabiliste
  % Necessité de plusieurs itérations: jamais obtention d'un état pur
% Implémentations: simulation sur ordinateur classique et implémentation quantique

%% Conclusion: Perspectives

% Application de la méthode de newton ?
% 

%====================================
%====================================

% 4 minutes
\section{Introduction}

% 2 minutes
\begin{frame}
\frametitle{Etat d'un système quantique}

\begin{definition}
  Notation d'un état quantique :
\begin{equation}
    \ket{\psi} = \displaystyle\sum_{i} c_i \ket{k_i},
\end{equation}
Avec $\ket{k_i}$ vecteurs d'état purs.
\end{definition}

% Parler d'un espace de Hilbert (espace vectoriel muni de produit scalaire, et autres => à regarder)
% 
% Rajouter un exemple

Système quantique élémentaire: le qubit : $\ket{\psi} = \alpha \ket{0} + \beta \ket{1}$

\begin{example}
\begin{equation}
  \ket{\psi_1} = \frac{1}{\sqrt{2}} \ket{0} + \frac{1}{\sqrt{2}} \ket{1}.
\end{equation}
avec
\begin{equation}
  \ket{0} \mapsto \begin{bmatrix}
    1 \\ 0
  \end{bmatrix}
  ,
  \ket{1} \mapsto \begin{bmatrix}
    0 \\ 1
  \end{bmatrix}
\end{equation}
\end{example}

\end{frame}

% 1 minute
\begin{frame}
\frametitle{La mesure projective}
\begin{definition}
  Quand un système quantique est dans un état superposé $\ket{\psi} = \displaystyle\sum_{i} c_i \ket{k_i}$, on va avoir comme probabilité $c_i^2$ de mesurer l'état $\ket{k_i}$.
\end{definition}

\begin{rem}
  La mesure est \textbf{projective}: on pert l'état probabiliste.
\end{rem}
\end{frame}



% 1 minute
\begin{frame}
\frametitle{Dynamique des systèmes quantiques}
\begin{enumerate}
  \item Evolution unitaire: conservation de la norme
  \item Evolution linéaire: limite dans les problèmes résolvables
\end{enumerate}
\end{frame}

\begin{frame}
  \frametitle{Vers l'informatique quantique}
  \begin{itemize}
    \item Utiliser de multiples qubits - intrication quantique
    
    \begin{align}
      \ket{\psi_1} &= \frac{1}{2}\ket{00} + \frac{1}{2}\ket{01} + \frac{1}{2}\ket{10} +\frac{1}{2}\ket{11} \nonumber \\
                   &= \frac{1}{2} (\ket{0} + \ket{1}) \otimes (\ket{0} + \ket{1})  \nonumber
    \end{align}
    \begin{align}
      \ket{\psi_2} &= \frac{1}{4}\ket{00} + 0\ket{01} + 0\ket{10} +\frac{1}{4}\ket{11} \nonumber \\
                   &= ?? \nonumber
    \end{align}

    \item Portes quantiques
    \begin{itemize}
      \item Portes élémentaires: Hadamard, Pauli, Toffoli
      \item Construction de circuit
    \end{itemize}
  \end{itemize}
\end{frame}

%====================================
%====================================

% 6 minutes
\section{Algorithme de Deutsch-Jozsa}

% 2 minutes
\begin{frame}
\frametitle{Présentation du problème}

\begin{pb}
  Déterminer en le moins d'itérations possibles si une fonction $f$ booléenne est constante ou équilibrée
\end{pb}

\medbreak
Dans le cas classique: $2^{n-1} + 1$ itérations

\medbreak
Dans le cas quantique: 1 seule itération

\end{frame}

% 3 minutes
\begin{frame}
\frametitle{Algorithme}

\begin{figure}[htbp]
  \centering
  \centerline{
      \Qcircuit @C=1em @R=.7em {
        & \lstick{\ket{0}^n} \barrier[-1.75em]{1} & \gate{H^{\otimes n}} \barrier[-1.5em]{1} & \multigate{1}{U_f} \barrier[-1.75em]{1} & \gate{H^{\otimes n}} \barrier[-1.5em]{1} & \meter \\
        & \lstick{\ket{1}} & \gate{H} & \ghost{U_f} & \qw & \\
        \hspace{3em} \ket{u_{0}} & \hspace{9em} \ket{u_{1}} &  \hspace{10em} \ket{u_{2}} & \hspace{10em} \ket{u_{3}}
      }
  }
  \label{fig:univerise}
\end{figure}

\begin{enumerate}
  \item Initialisation: $\ket{u_0}$
  \item $\ket{u_1}$: Mise à l'équilibre: porte de Hadamard
  \item $\ket{u_2}$: Application de la fonction $U_f$
  \item $\ket{u_3}$: Préparation pour la mesure
\end{enumerate}

\end{frame}

% 2 minutes
\begin{frame}[fragile]
\frametitle{Implémentation en \texttt{python (Qiskit)}}

\begin{lstlisting}[style=CStyle]
import numpy as np
from qiskit import *
import matplotlib.pyplot as plt
nb_qubits = 3
circuit = QuantumCircuit(nb_qubits, nb_qubits)
for i in range(nb_qubits):
    circuit.h(i)
circuit.barrier()
circuit.z(0)
circuit.z(2)
circuit.barrier()
for i in range(nb_qubits):
    circuit.h(i)
circuit.barrier()
for i in range(nb_qubits):
    circuit.measure(i, i)
\end{lstlisting}

\end{frame}

%====================================
%====================================

% 6 minutes
\section{Algorithme de Grover}

% 2 minutes
\begin{frame}
\frametitle{Présentation du problème}
\end{frame}

% 3 minutes
\begin{frame}
\frametitle{Itération unique}
\end{frame}

% 2 minutes
\begin{frame}
\frametitle{Implémentations classiques et quantiques}
\end{frame}

%====================================
%====================================

% 2 minutes
\section{Conclusions}

\begin{frame}
\frametitle{Le qubit}
\end{frame}

\begin{frame}
\frametitle{Biliographie}
\begin{itemize}
    \fontsize{6pt}{7.2}\selectfont

    \item R. Buyya, R. N. Calheiros, J. Son, A. V. Dastjerdi and Y. Yoon, "Software-Defined Cloud Computing: Architectural elements and open challenges," 2014 International Conference on Advances in Computing, Communications and Informatics (ICACCI), New Delhi, 2014, pp. 1-12
    \item Bhore, Pratik. (2016). A Survey on Storage Virtualization and its Levels along with the Benefits and Limitations. INTERNATIONAL JOURNAL OF COMPUTER SCIENCES AND ENGINEERING. 4. 115-121.
    \item M. F. Bari et al., "Data Center Network Virtualization: A Survey," in IEEE Communications Surveys \& Tutorials, vol. 15, no. 2, pp. 909-928
    \item M. Alouane and H. El Bakkali, "Virtualization in Cloud Computing: Existing solutions and new approach," 2016 2nd International Conference on Cloud Computing Technologies and Applications (CloudTech), Marrakech, 2016, pp. 116-123
    \item Chrobak, Pawel. (2014). Implementation of Virtual Desktop Infrastructure in academic laboratories. 1139-1146.
    \item Nagesh, O \& Kumar, Tapas \& Venkateswararao, V.. (2017). A Survey on Security Aspects of Server Virtualization in Cloud Computing. International Journal of Electrical and Computer Engineering. 7. 1326-1336.
\end{itemize}

\end{frame}

\end{document}
