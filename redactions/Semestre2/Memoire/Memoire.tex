\documentclass[12pt,a4paper]{report}

\usepackage{mathtools}
\usepackage{graphicx}
\usepackage{braket}
\usepackage{amsthm}
\usepackage{amsfonts}
\usepackage{lmodern}
\usepackage[utf8]{inputenc}
\usepackage[french]{babel}
\usepackage[T1]{fontenc}
\usepackage{subcaption}
\usepackage{caption}
\usepackage{gensymb}
\usepackage{tikz}
\usepackage{qcircuit}
\usepackage{listings}
\usepackage{pgfplots}
\usepackage{appendix}
\usepackage{hyperref}
\usepackage[ruled,vlined]{algorithm2e}
\usepackage{titlesec}
\usepackage{cleveref}
\usepackage{float}
\definecolor{gray75}{gray}{0.75}
\DeclareMathOperator{\tr}{tr}
\newcommand{\hsp}{\hspace{20pt}}
\titleformat{\chapter}[display]
{\normalfont%
    \Large% %change this size to your needs for the first line
    \bfseries}{\chaptertitlename\ \thechapter}{15pt}{%
    \Large %change this size to your needs for the second line
    }
\usetikzlibrary{angles,quotes}
% \usepackage{a4wide}

\definecolor{main-color}{rgb}{0.6627, 0.7176, 0.7764}
\definecolor{back-color}{rgb}{0.95,0.95,0.92}
\definecolor{mGreen}{rgb}{0,0.6,0}
\definecolor{string-color}{rgb}{0.3333, 0.5254, 0.345}
\definecolor{key-color}{rgb}{0.8, 0.47, 0.196}

\lstdefinestyle{CStyle}{
  language = C++,
  basicstyle=\tiny,
  backgroundcolor = {\color{back-color}},
  commentstyle=\color{mGreen},
  stringstyle = {\color{string-color}},
  keywordstyle = {\color{key-color}},
  keywordstyle = [2]{\color{blue}},
  keywordstyle = [3]{\color{gray}},
  otherkeywords = {;,<<,>>,++},
  morekeywords = [2]{function,end,Constants,Variables,Minimize,Constraints},
  morekeywords = [3]{tr,EntropyM,EntropyP,EntropyMP,MutualInformation,I,xlog},
  breakatwhitespace=false,
  breaklines=true,
  captionpos=b,
  keepspaces=true,
  numbers=left,
  numbersep=5pt,
  showspaces=false,
  showstringspaces=false,
  showtabs=false,
  tabsize=2,
}


\makeatletter
\newcommand\frontmatter{
    \cleardoublepage
    \pagenumbering{roman}
}
\newcommand\mainmatter{
    \cleardoublepage
    \pagenumbering{arabic}
}
\newcommand\backmatter{
  \if@openright
    \cleardoublepage
  \else
    \clearpage
  \fi
}
\newcommand{\tens}[1]{%
  \mathbin{\mathop{\otimes}\limits_{#1}}%
}

\makeatother
\newcommand{\norm}[1]{\left\lVert#1\right\rVert}
\newcommand{\Mod}[1]{\ (\mathrm{mod}\ #1)}
\graphicspath{{images/}}

\newtheorem{definition}{Définition}
\newtheorem{pb}{Problème}
\newtheorem{rem}{Remarque}
\newtheorem{ex}{Exemple}
\newtheorem{theoreme}{Théorème}

\begin{document}
\begin{titlepage}
    \begin{center}
      \begin{figure}[!tbp]
        \centering
        \begin{subfigure}[b]{0.3\textwidth}
          \includegraphics[width=\textwidth]{Polytech_Angers.png}
        \end{subfigure}
        \hfill
        \begin{subfigure}[b]{0.3\textwidth}
          \includegraphics[width=\textwidth]{LogoUnivAngers.png}
        \end{subfigure}
        \hfill
        \begin{subfigure}[b]{0.2\textwidth}
          \includegraphics[width=\textwidth]{LogoLARIS.png}
        \end{subfigure}
      \end{figure}
      {\large Master Systèmes Dynamiques et Signaux}\\[0.5cm]
      {\large Mémoire de master}\\[0.5cm]
      \rule{\linewidth}{0.5mm} \\[0.4cm]
      { \huge \bfseries Conception de détecteurs quantiques optimaux \\ via le calcul par intervalles \\[0.4cm] }
      \rule{\linewidth}{0.5mm} \\[1.5cm]
      \noindent
      \begin{minipage}{0.4\textwidth}
        \begin{flushleft} \normalsize
          \emph{Auteur :}\\
          M. Pierre \textsc{Engelstein}\\
          \end{flushleft}
          \end{minipage}%
          \begin{minipage}{0.4\textwidth}
          \begin{flushright} \normalsize
          \emph{Encadrants :}\\
          Dr. Nicolas \textsc{Delanoue}\\
          Pr. François \textsc{Chapeau-Blondeau}\\
          \end{flushright}
      \end{minipage}
      \vfill
      \normalsize
      \emph{Jury :}
      \begin{tabular}{lc}
          Pr.~Laurent \textsc{Hardouin}\\
          Dr.~Nicolas \textsc{Delanoue}\\
          Pr.~François \textsc{Chapeau-Blondeau}\\
          Pr.~Sébastien \textsc{Lahaye}\\
          Dr.~Mehdi  \textsc{Lhommeau}\\
          Pr.~David  \textsc{Rousseau}\\
      \end{tabular}

      \vspace{2cm}

      % Version du \\ \today
      % {2020-2021}
    \end{center}
\end{titlepage}

\frontmatter

\section*{Remerciements}
Je remercie Dr. Nicolas Delanoue et Pr. François Chapeau-Blondeau pour leur encadrement sur ce travail. Je remercie également mes parents pour les encouragements et l'aide apportés sur cette année de Master.

\pagebreak
\tableofcontents

\pagebreak
\listoffigures

\mainmatter

\chapter{Introduction}

L'informatique quantique, application des théories quantiques développées depuis le début du vingtième siècle à la théorie de l'information puis spécifiquement au calcul, est aujourd'hui en plein développement, théorique avec la découverte de nouveaux algorithmes, mais aussi pratique avec l'intérêt porté par les différents industriels. On voit alors l'apparition de nouvelles plateformes basées sur des processeurs quantiques, permettant de mettre en place les différentes avancées théoriques.

L'informatique quantique offre une accélération de certains traitements permettant théoriquement d'effectuer des calculs qui seraient infaisables en des temps raisonnables sur nos calculateurs classiques, par exemple le cassage des clés cryptographiques dans l'ordre de minutes au lieu de dixaines d'années.

Avec ce nouveau champ, de nombreuses questions se posent sur nos infrastructures actuelles, notamment en termes de cybersécurité et il est alors important de comprendre les capacités qu'offre l'informatique quantique.

Ce travail présente en première partie les notions fondamentales à la compréhension de l'informatique quantique. On y montre ce qu'est un qubit, puis on explique le mécanisme de mesure qui vient apporter de la probabilité, et enfin les principes d'évolution de systèmes quantiques permettant de construire des systèmes de calcul.

La deuxième partie de ce travail présente un tableau des mises en \oe{}uvres expérimentales au travers des processeurs quantiques et des simulateurs, développés par les différents industriels tels que Microsoft, IBM, Google et Atos depuis le début des années 2010.

Enfin, la dernière partie présente trois algorithmes majeurs au développement de l'informatique quantique, qui notamment illustrent les apports spécifiques du quantiqe, avec des performances de traitement de l'information inaccessibles en classique. On explique tout d'abord l'algorithme de Deutsch-Jozsa, sur la parallélisation d'évaluation de fonction; puis l'algorithme de Grover, sur la recherche de base de données; et enfin l'algorithme de Shor sur la factorisation en nombres premiers.
\chapter{Informatique quantique: éléments de base}


Les notions de base d'informatique quantique sont décrites dans plusieurs ouvrages de référence, notamment dans \cite{Nielsen00, Mermin07}. On présente ici un résumé des notions fondamentales à connaître pour la suite du rapport.

\section{Postulats de base}

On pose 3 postulats, servant de base aux raisonnements qui suivront. Ces postulats sont confirmés jusqu'à présent par les expériences.

\subsection{Etat d'un système quantique}
Un système quantique peut être représenté par un vecteur d'état, de la même manière qu'un système physique classique. On le représente par la notation de Dirac, notée de la forme $\ket{\psi}$. Ce vecteur d'état est nécessairement de norme 1 (la somme des modules au carré vaut 1). On peut distinguer deux types d'états pour un système quantique: les états de base, formant une base orthonormée d'un espace vectoriel complexe, et les états superposés. Ces états superposés correspondent à une combinaison linéaire des états de base. On peut écrire généralement un état quantique de la façon suivante:

\begin{equation}
    \ket{\psi} = \displaystyle\sum_{i} c_i \ket{k_i},
\end{equation}

avec les $\ket{k_i}$ états de base, et les $c_i$ respectant $ \displaystyle\sum_{i} |c_i|^2 = 1$ pour la normalisation du vecteur d'état.

\medbreak

Dans le cadre de l'informatique quantique, on utilise le système quantique le plus simple, appelé \textbf{qubit}. Ce système quantique est composé de deux états de base, $\ket{0}$ et $\ket{1}$, et des états superposés. Similairement à l'informatique classique, où on travaille sur le système physique le plus élémentaire - le bit - en quantique on travaille sur le système physique quantique élémentaire - le qubit. On dispose des mêmes états de base, mais l'informatique quantique apporte les états \textit{intermédiaires} superposés. Dans la base canonique $\{\ket{0}, \ket{1}\}$, on note un qubit de la façon suivante: $\ket{\psi} = \alpha \cdot \ket{0} + \beta \cdot \ket{1}$.

\subsection{Mesure projective}
Que se passe-t-il quand on mesure un système quantique ? On a évoqué au dessus la notion de superposition des états. L'expérience montre que, lorsqu'on va mesurer un système quantique, on va mesurer au hasard un des états de base, avec comme probabilité le carré du coefficient correspondant.

Mathématiquement, la mesure effectue une projection de l'état du système sur un des états de base dont il est composé. Par exemple, si on a un qubit dans l'état $\ket{\psi} = \frac{1}{\sqrt{2}}\ket{0} + \frac{1}{\sqrt{2}}\ket{1}$, alors la probabilité de mesurer 0, c'est-à-dire de projeter le système dans l'état de base $\ket{0}$ est $|\frac{1}{\sqrt{2}}|^2 = \frac{1}{2}$; et de même pour l'état de base $\ket{1}$. On a donc exactement la même probabilité de mesurer $\ket{0}$ que de mesurer $\ket{1}$.

Il faut noter que, lorsqu'on fait la mesure, on projette réellement le système quantique dans l'état de base. Concrètement, si on a un état superposé qu'on mesure, il se place dans l'état de base qu'on mesure, et toutes les mesures successives qu'on fera sur ce qubit donneront le même résultat. La mesure fait donc perdre l'état qu'on avait auparavant.

\subsection{Dynamique du système}
Comme n'importe quel système physique, on peut faire évoluer un système quantique dans le temps. Ici apparaissent deux propriétés. Tout d'abord, il découle du premier postulat que la dynamique d'un système quantique doit conserver la norme unité. En effet, un état quantique doit, pour être valide, avoir une norme de 1, et donc l'évolution d'un système quantique d'un premier état vers un autre doit conserver cette unitarité. Cela veut dire que la matrice représentant l'évolution du système quantique doit respecter la propriété suivante:

\begin{equation}
    U^{\dagger}U = UU^{\dagger} = I,
\end{equation}

avec $U$ la matrice d'évolution du système, $U^{\dagger}$ la matrice conjuguée transposée, ou adjointe, de $U$, et $I$ l'identité.

Une deuxième propriété est également posée, ne découlant pas des deux postulats précédents. La dynamique d'un système quantique doit être aussi linéaire. Ainsi, on pourrait penser que n'importe quelle évolution unitaire serait valable, mais l'expérience nous montre que non, il faut en plus qu'elle soit linéaire.

\section{Vers une informatique quantique}
\`A partir de ces 3 postulats de base, on peut commencer à comprendre comment se construit l'informatique quantique, et quels sont les apports sur l'informatique classique.

\subsection{Multiples qubits}
On a vu la définition d'un qubit. Cela nous permet d'étendre ce système quantique élémentaire à des systèmes composés de multiples qubits. En informatique classique, on travaille quasi systématiquement sur des mots binaires plutôt que des bits uniques; l'équivalent est vrai en quantique. Pour cela, les systèmes quantiques, et donc les qubits, sont munis d'une opération: le produit tensoriel. Quand on veut effectuer une combinaison de deux qubits, cela revient à faire un produit tensoriel des états des deux qubits individuels. Par exemple, si nous disposons de deux qubits ayant pour valeur $\ket{\psi_1} = \ket{0}$ et $\ket{\psi_2} = \ket{1}$, alors on peut écrire le 2-qubit combinaison des deux de la façon suivante:


\begin{equation}
    \ket{\psi} = \ket{0} \tens{} \ket{1},
\end{equation}

qu'on écrit généralement sous la forme plus simple:

\begin{equation}
    \ket{\psi} = \ket{01}.
\end{equation}

Prenons un 2-qubit formé par la combinaison de 2 qubits:

\begin{align*}
\ket{\psi} &= (\alpha_1\cdot\ket{0} + \beta_1\cdot\ket{1}) \otimes (\alpha_2\cdot\ket{0} + \beta_2\cdot\ket{1}) ,\\
&= \alpha_1\alpha_2 \ket{0} \otimes \ket{0} + \alpha_1\beta_2 \ket{0} \otimes \ket{1} + \beta_1\alpha_2 \ket{1} \otimes \ket{0} + \beta_1\beta_2 \ket{1} \otimes \ket{1}.
\end{align*}

On peut donc, si on a un 2-qubit combinaison linéaire de tous les états de base, le factoriser en deux qubits séparés qu'on peut caractériser.

\medbreak

Considérons maintenant le 2-qubit suivant:

\[
\ket{\psi} = \gamma_1 \ket{00} + \gamma_2 \ket{11}
\]

Il parait évident alors qu'on ne peut pas factoriser ce 2-qubit en produit tensoriel de 2 qubits individuels. Dans ce cas, on dit que les deux qubits sont \textbf{intriqués} et donc non séparables.


\subsection{Portes quantiques}
Dans la représentation d'état classique, et spécifiquement en informatique, on peut faire évoluer l'état au travers de portes. En informatique classique, on dispose ainsi de portes élémentaires telles que \texttt{AND}, \texttt{NOT}, \texttt{OR}, etc. 

De la même manière, en respectant le troisième postulat posé précédement, on peut construire des portes logiques quantiques, les combiner, afin de créer des circuits quantiques. Ces portes quantiques sont nécessairement unitaires, donc inversibles. En informatique quantique, on distingue donc plusieurs portes élémentaires, utilisées dans beaucoup de circuit \cite{Barenco95}:

Les portes quantiques sont complètement caractérisées par la façon dont elles transforment les états quantiques dans la base canonique. On peut alors utiliser des tables de vérité pour les définir, de la même façon qu'en informatique classique:

\begin{enumerate}
    \item La porte de Hadamard $H$. Elle permet de passer un qubit d'un état de base $\ket{0}$ à l'état superposé $\frac{1}{\sqrt{2}}\ket{0} + \frac{1}{\sqrt{2}}\ket{1}$, ou de l'état de base $\ket{1}$ à l'état superposé $\frac{1}{\sqrt{2}}\ket{0} - \frac{1}{\sqrt{2}}\ket{1}$. Elle est très utilisée en début de circuit pour préparer les qubits entrants dans un état permettant l'évaluation parallèle de toutes les entrées;
    \item Les portes de Pauli $X$, $Y$ et $Z$ permettant d'effectuer des rotations aux états des qubits;
    \item La porte de Toffoli, similaire d'un \texttt{NON} booléen à 3 qubit (il effectue un \texttt{NON} sur le dernier qubit quand les deux premiers sont à $\ket{1}$), est une porte universelle quantique \cite{shi2002toffoli}. Elle permet donc de construire l'ensemble des autres portes faisables.
\end{enumerate}

Les tables de vérité des différentes portes quantiques évoquées sont disponibles en annexes.

Un exemple de circuit est le suivant: On dispose d'un 3-qubit dans l'état $\ket{000}$. Au départ, on applique à ces trois qubits une porte de Hadamard, qui les fait se retrouver dans une superposition équilibrée des états de base (c'est-à-dire de sorte qu'une mesure nous donne l'un des états de base avec une probabilité de $\frac{1}{8}$). On applique ensuite deux portes de Pauli $Z$, un au premier qubit, et un au troisième. On applique de la même façon 3 portes de Hadamard à la sortie, puis on mesure.
\pagebreak
\begin{figure}[t!]
    \centerline{
        \Qcircuit @C=1em @R=.7em {
            & \lstick{\ket{0}} & \gate{H} \barrier[-1.25em]{2} & \gate{Z} \barrier[-1.25em]{2} & \gate{H} & \meter & \qwa \\
            & \lstick{\ket{0}} & \gate{H} & \qw & \gate{H} & \meter & \qwa \\
            & \lstick{\ket{0}} & \gate{H} & \gate{Z} & \gate{H} & \meter & \qwa
        }
    }
    \caption{Exemple de circuit quantique, avec $U_f$ pour $f(a, b, c) = (\neg{a})\oplus c$}
\end{figure}

Dans cet exemple, les deux portes de Pauli Z permettent d'appliquer une fonction $f$ booléenne. En informatique classique, pour évaluer cette fonction sur les $2^n$ entrées possibles, il faudrait effectuer $2^n$ évaluations de $f$. Ici, avec ce circuit quantique, on n'effectue qu'une seule évaluation quantique de $f$ sur l'état équilibré donné par les portes de Hadamard. On vient évaluer la fonction $f$ sur tout les états de base composant l'état quantique. Puisque cet état est équilibré, les $2^n$ états de base vont être évalués, c'est-à-dire toutes les entrées qu'on voulait évaluer en classique.


L'avantage du quantique bien montré ici: les deux portes du milieu vont faire changer l'état des qubits, mais en parallèle: on fait évoluer le système simultanément pour tous les états de base qui nous intéressent, puisqu'ils sont superposés.

\chapter{Calcul par intervalles: éléments de base}

Le calcul par intervalles est décrit au départ dans les travaux de Ramon Moore \cite{Moore66}. L'utilité de ce mode de calcul vient des problèmes que représente le stockage des nombres réels dans nos ordinateurs via la norme \texttt{IEEE 754}. En effet, on sait avec cette norme facilement représenter une certaine quantité, finie, de nombres réels tels que $0.5$, sous la forme $\texttt{signe} \times \texttt{base}^{\texttt{exposant}} \times (1 + \texttt{mantisse})$. Il est en revanche impossible de représenter exactement la plupart des nombres réels, tels que $0.1$. De ce fait, lorsqu'on se place dans des contextes de calculs, on peut se retrouver à accumuler des erreurs de précision qui vont venir fausser les résultats. Quand on veut garantir des résultats, par exemple sur des problèmes d'optimisation, cela peut devenir pénalisant.

Le calcul par intervalles permet en outre de pouvoir caractériser l'ensemble des solutions d'un problème, d'en obtenir une caractérisation globale. Cela permet de garantir qu'on a bien l'optimum global sur tout l'ensemble des solutions admissibles d'un problème d'optimisation. Dans ce chapitre, on présente les notions formant la base du calcul par intervalle ainsi qu'un algorithme d'optimisation utilisant cette méthode de calcul.


\section{Les intervalles}
\subsection{Intervalle et boite}

\begin{definition}
  On définit un \textit{intervalle} $[\underline{x}, \overline{x}]$ comme l'ensemble des nombres réels $x$ tels que $\underline{x} \leq x \leq \overline{x}$.
\end{definition}

On note par la suite plus généralement $[x] = [\underline{x}, \overline{x}]$.

\begin{ex}
  Si on veut représenter le nombre $\sqrt{2} = 1.4142\dots$, on peut dire: $1.4 \leq \sqrt{2} \leq 1.5$, donc encadrer ce nombre par l'intervalle $[1.4, 1.5]$.
\end{ex}

On étend cette notion d'intervalle à plusieurs variables en prenant le produit cartésien de plusieurs intervalles pour former des boites en $n$ dimensions:

\begin{definition}
  Une \textit{boite} $[\textbf{x}]$ est le produit cartésien des intervalles qui composent la boite : $[\textbf{x}] = [x_1] \times [x_2] \times \dots \times [x_n]$.
\end{definition}

\subsection{Fonction d'inclusion}
% Fonction d'inclusion naturelle
Avec cette notion d'intervalle, on peut définir le comportement quand on applique une fonction. L'idée est de se dire que, pour un intervalle ou une boite d'entrée $[x]$, l'intervalle image par une fonction $f$ doit contenir l'ensemble des images prises par la fonction $f$ pour tout les $x \in [x]$ :

\begin{definition}
    Soit $f : \mathbb{R}^n \rightarrow \mathbb{R}^m$ une fonction, la fonction $[f] : \mathbb{R}^n \rightarrow \mathbb{R}^m$ est une \textbf{fonction d'inclusion} pour $f$ si

    \begin{align}
        \forall[x] \in \mathbb{R}^n , f([x]) \subset [f]([x])
    \end{align}
\end{definition}

\begin{ex}
  La figure \ref{fig:fct2} montre l'encadrement d'une fonction $y = f(x)$ quelconque.

  \begin{figure}[H]
    \centering
    \includegraphics[scale=0.5]{intervaleval/function_eval_2.png}
    \caption{Fonction d'inclusion}
    \label{fig:fct2}
  \end{figure}
\end{ex}

Les fonctions d'inclusion nous intéressant ici sont celles convergentes, c'est-à-dire celles dont la taille de la boite image tend vers 0 quand la taille de la boite d'entrée tend vers 0.

\begin{definition}
  Une fonction d'inclusion $[f]$ de $f$ est dite \textbf{convergente} si 

  \begin{align}
    w([x]) \rightarrow 0 \Rightarrow w([f]([x])) \rightarrow 0,
  \end{align}

  Avec $w([x])$ fonction diamètre de la boite $[x]$.
\end{definition}

% On impose également que la fonction d'inclusion $[f]$ vérifie $[f]({x}) = {f(x)}$, c'est-à-dire qu'elle soit convergente. On impose ainsi que la fonction $[f]$ soit continue au voisinage de ces intervalles "points". 

\medbreak

L'encadrement nécessite la connaissance précise de la forme de la fonction pour pouvoir l'encadrer correctement. Ceci peut se révéler compliqué pour des fonctions non-évidentes, typiquement quand on monte en dimension. Pour cela, on peut combiner les fonctions d'inclusion sans perdre la garantie d'inclusion \cite{neumaier91, Delanoue18} , comme indiqué dans le Théorème \ref{def:circ}.

\begin{theoreme}
  \label{def:circ}
  Si $[f]$ et $[g]$ sont des fonctions d'inclusion respectives pour $f$ et $g$, alors $[f] \circ [g]$ est une fonction d'inclusion pour $f \circ g$.
\end{theoreme}

Cela permet en pratique de construire des fonctions d'inclusions élémentaires puis de les combiner. En effectuant cette opération, on peut en revanche perdre de la précision sur l'encadrement comme le montre la figure \ref{fig:fct3}.

\begin{figure}[H]
  \centering
  \includegraphics[scale=0.5]{intervaleval/function_eval_3.png}
  \caption{Fonction d'inclusion composée de moindre qualité}
  \label{fig:fct3}
\end{figure}

% Fonction d'inclusion par la forme centrée

\subsection{Arithmétique élémentaire}

Comme dit précédemment, on peut construire les fonctions d'inclusions des fonctions nécessaires pour n'importe quel problème en calcul par intervalle. Spécifiquement, il est utile de définir un certain nombre de fonctions de base permettant de former les briques de construction pour la formation de fonctions composées. On peut ainsi définir les opérateurs binaires (l'addition, la soustraction, la multiplication, \dots) ainsi que les opérateurs unaires (l'exponentielle, la puissance, le sinus, \dots).

\begin{ex}
  
  Un certain nombre de fonctions arithmétiques élémentaires peuvent être formulées , avec $[x_1] = [\underline{x_1}, \overline{x_1}]$ et $[x_2] = [\underline{x_2}, \overline{x_2}]$:
  \begin{itemize}
    \item $[x_1] + [x_2] = [\underline{x_1} + \underline{x_2}, \overline{x_1} + \overline{x_2}]$
    \item $[x_1] - [x_2] = [\underline{x_1} - \overline{x_2}, \overline{x_1} - \underline{x_2}]$
    \item $[x_1] \times [x_2] = [\text{min}(\underline{x_1}\underline{x_2}, \underline{x_1}\overline{x_2}, \overline{x_1}\underline{x_2}, \overline{x_1}\overline{x_2}), \text{max}(\underline{x_1}\underline{x_2}, \underline{x_1}\overline{x_2}, \overline{x_1}\underline{x_2}, \overline{x_1}\overline{x_2})]$
    \item $e^{[x]} = [e^{\underline{x}}, e^{\overline{x}}]$
    \item \dots
  \end{itemize}
\end{ex}

On peut étendre ces définitions à l'ensemble des fonctions strictement monotones: il est évident de se dire que, si une fonction $f(x)$ est strictement croissante, alors $[f]([x]) = [f(\underline{x}), f(\overline{x})]$ est une fonction d'inclusion pour $f$. On peut alors construire des fonctions moins évidentes, comme $f : x \mapsto x^3$ en découpant la définition de la fonction par morceaux monotones.

\section{Optimisation avec les intervalles}
On met en place un algorithme d'optimisation utilisant le calcul par intervalle pour obtenir un encadrement garanti de la solution à notre problème.

On veut résoudre le problème $\max(f(x))$ tel que $g(x) \leq 0$, avec $f$ fonction coût et $g$ un ensemble des contraintes. Avec le calcul par intervalles, on cherche à avoir un encadrement \textit{garanti}, \textit{global} de la solution au problème. Le principe de base est de découper l'ensemble des entrées en un certain nombre de boites, dépendant de la précision que l'on veut, comme à la figure \ref{fig:optim1}. On choisit ensuite un $a$ solution admissible du problème suivant le théorème \ref{thm:solutionsup}.

\begin{theoreme}
  \label{thm:solutionsup}
  Soit un $a$ une solution admissible du problème $\max\limits_{x} f(x)$ tel que $g(x) \leq 0$ et $x^*$ la solution optimale, on a:
  \begin{align}
      sup([f]([x])) \leq f(a) \Rightarrow x^* \notin [x]
  \end{align}
\end{theoreme}

Cela nous permet d'éliminer directement de l'ensemble des solutions les boites dont la borne supérieure de l'image est inférieure à l'image de ce candidat $a$, puisque garanties comme ne contenant pas l'optimum du problème. La figure \ref{fig:optim2} illustre cette élimination. \`A l'issue de cette étape, on voit qu'on obtient un ensemble plus restreint de boites garanties comme contenant la solution, et on peut itérer en choisissant au fur et à mesure un candidat $a$ meilleur, et on arrive à un encadrement satisfaisant de la solution comme à la figure \ref{fig:optim3} avec l'intervalle $[x]^*$.

\begin{figure}[H]
  \centering
  \begin{subfigure}[h]{0.3\textwidth}
      \centering
      \includegraphics[scale=0.4]{AlgoOptim/function_optim_1.png}
      \caption{}
      \label{fig:optim1}
  \end{subfigure}
  \begin{subfigure}[h]{0.3\textwidth}
      \centering
      \includegraphics[scale=0.4]{AlgoOptim/function_optim_4.png}
      \caption{}
      \label{fig:optim2}
  \end{subfigure}
  \begin{subfigure}[h]{0.3\textwidth}
      \centering
      \includegraphics[scale=0.4]{AlgoOptim/function_optim_6.png}
      \caption{}
      \label{fig:optim3}
  \end{subfigure}
  \caption{Optimisation naïve}
\end{figure}

Cette méthode d'optimisation, "naïve", permet d'obtenir un résultat satisfaisant, mais va être rapidement limitée si on veut des précisions plus élevées. En effet, on va avoir très rapidement un très grand nombre de boites à traiter. On peut remarquer entre autres qu'un certain nombre de boites vont être dans des "régions" globales pouvant être éliminées (par exemple, sur la figure \ref{fig:optim2}, on a tout le bas de la fonction qui pourrait être éliminé d'un coup). Cela nous amène à un algorithme plus avancé, présenté en \ref{fig:algomaxim}.

Au lieu de découper directement l'espace des entrées en un très grand nombre de boites, on va bisecter l'espace en deux boites, suivant l'axe le plus grand. On obtient deux boites, dont l'axe le plus grand aura été coupé en faisant $[x] \longrightarrow \{[\underline{x}; (\overline{x} + \underline{x}) . 0.5], [(\overline{x} + \underline{x}) . 0.5; \overline{x}]\}$. On considère la bissection par le milieu, mais on pourrait aussi utiliser des bissections plus avancées permettant d'accélérer l'algorithme.

Une fois ces deux boites obtenues, on peut évaluer la fonction d'inclusion et les contraintes, et décider de supprimer les boites ne rentrant pas dans les contraintes. Sur l'ensemble des boites obtenues pour une itération, on effectue l'opération de recherche de critère décrit précédemment, et on réitère sur les boites restantes. Cela permet d'éliminer rapidement les boites de plus grande taille qui sont garanties ne contenant pas l'optimum, et donc réduire considérablement le nombre de boites à traiter par la suite.

On considère l'algorithme fini quand on a obtenu une précision suffisante sur l'encadrement de la fonction ou des variables d'entrée, ou alors quand un certain nombre d'itérations ont été effectuées.

\section{Implémentation}

L'algorithme \ref{fig:algomaxim} a été implémenté en C\# (dotnet 5) sur la base de la librairie \href{https://github.com/selmaohneh/IntSharp}{IntSharp} modifiée pour répondre à nos besoins (rajout de l'intervalle vide, de la fonction $x \longrightarrow x\log(x)$, des intervalles booléens, \dots). Une interface graphique basique utilisant Blazor a été mise en place pour faciliter la visualisation de l'optimisation et des différents problèmes rencontrés lors du développement. L'ensemble du projet est disponible sur \url{https://github.com/PierreEngelstein/IntervalEval}. La solution est organisée en plusieurs modules:

\begin{itemize}
  \item \texttt{IntervalEval} qui fourni la librairie de base pour le calcul par intervalle et l'optimisation;
  \item \texttt{IntervalEval.Front} est l'interface web basique développée avec le framework Blazor Server;
  \item \texttt{IntervalEval.Optimizer} est une interface en ligne de commande pour le problème spécifique détaillé dans ce rapport;
  \item \texttt{IntervalEval.FrontConsole} est une interface en ligne de commande codée pour tester le problème à trois états quantiques d'entrée, pour tester les performances sur un problème en plus haute dimension;
  \item \texttt{IntervalEval.Tests} fournit un ensemble de tests unitaires pour le bon fonctionnement de la librairie d'intervalle.
\end{itemize}


\begin{figure}[H]
    \begin{algorithm}[H]
        \SetAlgoLined
        \KwData{
          $[I_{init}]$ initial search box;
          $\epsilon$ stop criterion;
          $f$ cost function;
          $g$ constraints function;
        }
        \KwOut{
          $[f]$ bounds of best solution; $[I]$ solution box
        }
        \Begin{
          $\texttt{solutions}$ list of solution boxes\;
          Add $[I_{init}]$ to $\texttt{solutions}$\;
          $[f_{c}]$ current bounds of solutions\;
          \While{$\overline{f_{c}} - \underline{f_{c}} \geq \epsilon$}{
              $\texttt{currentSolutions}$ empty list of boxes\;
              \tcc{Bisect, evaluate cost, manage constraints}
              \ForAll{$\texttt{sol}$ in $\texttt{solutions}$}{
                    $[\texttt{left}], [\texttt{right}] \longleftarrow \texttt{bisect}(\texttt{sol})$\;
                    \If{$\texttt{g}([\texttt{left}])$ is valid}{
                        Add $[\texttt{left}]$ to $\texttt{currentSolutions}$\;
                    }
                    \If{$\texttt{g}([\texttt{right}])$ is valid}{
                        Add $[\texttt{right}]$ to $\texttt{currentSolutions}$\;
                    }
              }
              \tcc{Remove boxes certified not to contain maximum}
              Evaluate $[f]$ for all $[\texttt{currentSolutions}]$\;
              $f_{best}$ best $f(\texttt{sol}.\texttt{mid})$ in all $[f]$ \;
              Remove all $[\texttt{sol}]$ in $[\texttt{currentSolutions}]$ where $sup([f]([\texttt{sol}])) \leq f_{best}$\;
              $\texttt{solutions} \longleftarrow \texttt{currentSolutions}$
          }
          \Return{$\texttt{solutions}, [f_{c}]$}
        }
      \end{algorithm}

    \caption{Algorithme de maximisation par le calcul par intervalles}
    \label{fig:algomaxim}
\end{figure}
\chapter[Construction d'un détecteur quantique optimal]{Construction d'un détecteur quantique \\ optimal}

On présente ici le détail du problème de la détection optimale quantique avec le critère de l'information mutuelle. Les résultats présentés ici ont fait l'objet d'une proposition de communication à la journée "Traitement du signal et applications quantiques" du GdR CNRS ISIS \cite{engelstein21}.

\section{Formulation du problème}
Le problème de la détection d'état quantique porte sur un ensemble de $m$ états quantiques représentés par les opérateurs densité $\{\rho_j \; , \; 1 \leq j \leq m\}$ munis des probabilités à priori $\{p_j \geq 0 \; , \; 1 \leq j \leq m \}$. L'objectif est d'obtenir un ensemble de $n$ opérateurs de mesure $\{\Pi_k \; , \; 1 \leq k \leq n\}$ permettant d'identifier le mieux possible selon leur probabilités les états d'entrée qui nous arrivent.

En dimension 2, les opérateurs $\rho_j$ et $\Pi_k$ sont des matrices Hermitiennes semi-définies positives, de la forme $\begin{pmatrix}a & b+ic \\ b-ic & d \end{pmatrix}$.

Plusieurs critères ont été proposés à optimiser afin de construire ces détecteurs optimaux. D'une part, on a la possibilité de travailler sur la minimisation de l'erreur quadratique de mesure \cite{Eldar01} ou la maximisation de la probabilité de détection correcte \cite{Eldar03c}. D'autre part, et c'est ce sur quoi nous avons travaillé, on peut considérer le critère de l'information mutuelle entrée-sortie comme critère à maximiser \cite{Davies78}. Il s'agit d'un critère présentant une grande pertinence pour évaluer la transmission d'information sur des canaux de télécommunication. En particulier, l'information mutuelle permet de caractériser le débit maximal d'information qu'il est possible de transmettre sans erreur sur un canal de télécommunication donné.

\medbreak
L'information mutuelle de deux variables aléatoires $X$ et $Y$ a été formulée par Shannon en 1948 \cite{Shannon48}. Elle est donnée en fonction des distributions de probabilité $p_{(X, Y)}(x, y)$, $p_X(x)$ et $p_Y(y)$:

\begin{align}
    I(X;Y) = \displaystyle \sum_{y \in Y} \displaystyle \sum_{x \in X} p_{(X, Y)}(x, y) \log \big(\frac{p_{(X, Y)}(x, y)}{p_X(x) p_Y(y)}\big).
\end{align}

Elle peut aussi être écrite en fonction des entropies des variables aléatoires:

\begin{align}
    I(X; Y) &= H(X) - H(X | Y) \\
            &= H(Y) - H(Y | X) \\
            &= H(X) + H(Y) - H(X, Y).
\end{align}

Avec $H(X)$ entropie marginale de $X$, $H(Y)$ entropie marginale de $Y$, $H(X|Y)$ entropie conditionelle de $X$ sachant $Y$ et enfin $H(X, Y)$ entropie conjointe de $X$ et $Y$. On peut utiliser indifférement $\log_2$, $\log_{10}$ ou $\ln$ pour le logarithme, le changement étant à une constante multiplicative près.

Dans le cas classique, les entropies marginales, conditionnelles et conjointes sont définies par: 

\begin{align}
    H(X) &= -\displaystyle \sum_{x \in X} p_{X}(x) \log(p_{X}(x)) , \\
    H(Y) &= -\displaystyle \sum_{y \in Y} p_{Y}(y) \log(p_{Y}(y)) , \\
    H(X, Y) &= -\displaystyle \sum_{x \in X} \displaystyle \sum_{y \in Y} p_{(X, Y)}(x, y) \log(p_{(X, Y)}(x, y)), \\
    H(Y|X) &= -\displaystyle \sum_{x \in X, y \in Y} p_{(X, Y)}(x, y) \log \big(\frac{p_{(X, Y)}(x, y)}{p_{X}(x)}\big)
\end{align}

Dans le cas quantique, des formules spécifiques existent pour l'entropie d'un état quantique et l'information mutuelle, mais on ne les utilise pas spécifiquement ici. En effet, on s'intéresse à la transmission et récupération d'information classique, sur un canal quantique: on part d'information classique, les $X$, qu'on encode dans les $\rho_j$, pour de la transmition, et on mesure en quantique pour récupérer de l'information classique, les $Y$.

% Dans le cas quantique, les formules restent les mêmes, mais on exprime les probabilités des variables en fonction des valeurs des états quantiques d'entrée.

\medbreak

On a en entrée :

\begin{align}
    P(X = \rho_j) = p_j, \quad 1 \leq j \leq m.
\end{align}

La mesure quantique crée la distribution conditionnelle entrée sortie :

\begin{align}
    P(\Pi_k | X = \rho_j) = \tr(\rho_j \Pi_k) = \alpha_{jk}, \quad 1 \leq k \leq n,
\end{align}

avec $n \neq m$ possiblement. On peut en déduire la distribution de sortie:

\begin{align}
    P(Y = \Pi_k) &= \displaystyle \sum_{j = 1}^{m} P(\Pi_k | X = \rho_j)P(X = \rho_j) \\
    &= \displaystyle \sum_{j = 1}^{m} p_j \alpha_{jk}.
\end{align}

L'entropie de sortie est donc définie par: 

\begin{align}
    H(Y) &= -\displaystyle \sum_{k = 1}^{n} P(Y = \Pi_k) \log(P(Y = \Pi_k)) \\
    &= -\displaystyle \sum_{k = 1}^{n} \big(\displaystyle \sum_{j = 1}^{m} p_j \alpha_{jk}\big) \log\big(\displaystyle \sum_{j = 1}^{m} p_j \alpha_{jk}\big)
\end{align}

L'entropie d'entrée est elle définie par:

\begin{align}
    H(X) &= -\displaystyle \sum_{j = 1}^{m} P(X = \rho_j) \log(P(X = \rho_j)) \\
    &= -\displaystyle \sum_{j = 1}^{m} p_j \log(p_j)
\end{align}

La probabilité conjointe de $X$ et de $Y$ est donnée par: 

\begin{align}
    P(X = \rho_j, Y = \Pi_k) = p_j \tr(\rho_j \Pi_k),
\end{align}

Et donc l'entropie conjointe de $X$ et de $Y$ est donnée par:

\begin{align}
    H(X, Y) = \displaystyle \sum_{j=1}^{m} \displaystyle \sum_{k=1}^{n} p_j \tr(\rho_j \Pi_k).
\end{align}

On obtient donc l'information mutuelle entrée sortie dépendant à la fois des $p_j$ et des $\alpha_{jk}$ :

% On en déduit les probabilités marginales:

% \begin{align}
%     p(X = \rho_i) = \displaystyle \sum_{j}p_i \tr(\rho_i \Pi_j)  \\
%     p(Y = \Pi_j) = \displaystyle \sum_{i}p_i \tr(\rho_i \Pi_j),
% \end{align}

% Et les probabilités conditionelles:

% \begin{align}
%     P(Y=\Pi_j | X=\rho_i) = \frac{\tr(\rho_i \Pi_j)}{\displaystyle \sum_{k} \tr(\rho_i \Pi_k)}
% \end{align}

% L'information mutuelle pour notre problème peut donc être ré-écrite de la façon suivante, en utilisant $\alpha_{jk} = p_j\tr(\rho_j \Pi_k)$ :

\begin{align}
    \label{eq:mi}
    I(\rho; \Pi) &= H(\rho) + H(\Pi) - H(\rho, \Pi) \nonumber \\
    &= - \displaystyle \sum_{j=1}^{m} p_j \log \big( p_j \big) - \displaystyle \sum_{k=1}^{n} \big(\displaystyle \sum_{j=1}^{m} p_j\alpha_{jk} \big) \log \big( \displaystyle \sum_{j=1}^{m} p_j\alpha_{jk}\big) + \displaystyle \sum_{j=1}^{m} \displaystyle \sum_{k=1}^{n} p_j\alpha_{jk} \log( p_j\alpha_{jk} )
\end{align}

On peut aussi exprimer l'information mutuelle en fonction de l'entropie conditionnelle, mais il est plus efficace d'utiliser celle donnée à l'équation \ref{eq:mi} pour la résolution numérique.

Finalement, le problème se formule comme un problème de maximization de l'information mutuelle: on cherche les opérateurs de mesure $\Pi_k$ qui maximisent l'information mutuelle:

\begin{align}
    \max\limits_{\Pi} I(\rho, \Pi)
\end{align}
tel que :

\begin{align}
    \Pi_k \succeq 0, \quad 1 \leq k \leq n \label{eq:contrainte_sdp} \\
    \displaystyle \sum_{k=1}^{n} \Pi_k = I \label{eq:contrainte_somme_id}
\end{align}

La contrainte \ref{eq:contrainte_sdp} qui impose la semi-définie positivité des opérateurs de mesure $\Pi_k$. Enfin, la contrainte \ref{eq:contrainte_somme_id} permet d'obtenir des opérateurs de mesure cohérents pour que les probabilités de mesure $P(\Pi_k)$ soient positives et se somment à 1.

On est en présence d'une fonction non linéaire, convexe, et les contraintes engendrent un ensemble admissible convexe. C'est le cas idéal lors d'une minimisation, mais le problème est une maximisation, de même difficulté qu'une minimisation concave, on ne peut donc pas juste faire une descente de gradient pour le résoudre, l'optimum se situe sur la frontière. On peut utiliser un certain nombre de méthodes approximatives, nous utilisons le calcul par intervalle afin d'obtenir un encadrement global de la solution.

\section{Convexité de l'information mutuelle}
Davies considère dans \cite{Davies78} que l'information mutuelle pour ce problème peut être considérée comme étant convexe, simplifiant la résolution du problème en ayant à chercher le maximum sur les bords. On s'intéresse ici à l'étude de cette convexité.

Dans son article, Davies regroupe les traces et probabilités sous une seule variable $P_{jk} = p_j \tr(\rho_j \Pi_k)$. Ces coefficients $P_{jk}$ forment une matrice des probabilités, telle que :

\begin{align}
    \displaystyle \sum_{jk} P_{jk} = 1, \\
    \displaystyle \sum_{k}  P_{jk} = p_j.
\end{align}
L'information mutuelle s'écrit donc :

\begin{align}
    I(P) = \displaystyle \sum_{j} H(\displaystyle \sum_{k}P_{jk}) + \displaystyle \sum_{k} H(\displaystyle \sum_{j}P_{jk}) -  \displaystyle \sum_{jk} H(P_{jk}) 
\end{align}

La fonction $H(x) = -x \log(x)$ est convexe, et donc $I$ est convexe par rapport à la matrice des probabilités $P$. La figure \ref{fig:mi_convex} illustre cette fonction en fixant $p_1 = 0.3$ et $p_2 = 0.7$.

\begin{figure}[h]
    \centering
    \includegraphics[scale=0.2]{pb/MI_convex.png}
    \caption{Information mutuelle par rapport à la matrice de probabilités}
    \label{fig:mi_convex}
\end{figure}

La convexité semble bien vraie par rapport à $P$, mais on cherche à optimiser les matrices $\Pi_k$. La matrice $P$ comporte les traces de la multiplication $\rho_j \Pi_k$, qui est linéaire par rapport aux coefficients de $\Pi_k$. Si la fonction $I(P)$ est convexe par rapport à $P$, alors elle l'est par rapport aux $\Pi_k$, grace à la linéarité.

Quand on trace la même fonction, mais par rapport aux variables $\Pi_{\alpha_{jk}}$, en se fixant dans un espace deux dimensions, on s'aperçoit que la fonction n'est pas correctement définie sur les bords. Ceci est dû au fait que $x \longrightarrow x\log(x)$ n'est pas défini pour $x < 0$, ce qui fausse ou bloque les calculs, suivant l'implémentation.

\medbreak

\section{Formulation des contraintes}

La définition du problème permet de résoudre notamment les cas immédiats des opérateurs densité $\rho_j$ orthogonaux, mais la résolution devient très lente lorsqu'on passe à d'autres cas non orthogonaux. On rajoute des conditions au problème pour accélérer la résolution.

% Le premier élément à simplifier est l'expression de l'entropie marginale de $X=\rho_j$. En effet, nous l'avons exprimé en fonction de la trace de la multiplication matricielle, mais on peut reprendre la définition donnée lors du cas classique qui dit que $ H(X) = -\displaystyle \sum_{x \in X} p(x) \log(p(x))$. Le problème nous indique que nous connaissons les probabilités préalables des états d'entrée, on peut donc directement exprimer cette entropie en fonction de ces données et donc sans les variables de sortie $\Pi_i$.

Premièrement, on sait que les opérateurs de mesure se somment à l'identité. Cela signifie qu'on peut passer d'un problème à $n$ matrices à un problème à $n-1$ matrices pour $n \geq 2$. Les matrices étant carrées de dimension $N$, on passe de $n \times N^2$ variables à $(n - 1) \times N^2$ variables, ce qui est non négligeable.

De plus, le problème et les contraintes sont symétriques, une permutation des $\Pi_k$ n'influence pas le résultat de la fonction coût. Cela nous permet de couper le problème au moins en deux pour réduire à nouveau le temps de calcul. Du fait de la somme à l'identité, on peut ajouter en contrainte que $\Pi_{1_{1, 1}} \leq \frac{1}{n}$ pour $n$ opérateurs de mesure, puis $\Pi_{2_{1, 1}} \leq \frac{1}{n-1}$, etc.

Ensuite, on peut exprimer la semi-définie positivité des opérateurs de mesure en utilisant le critère de Sylvester. Dans le cas général, il indique que le déterminant de la matrice doit être positif ou nul, ainsi que les $n$ mineurs principaux. Dans le cas à deux dimension, cela se traduit par le déterminant et les deux éléments diagonaux postitifs ou nuls. On a ainsi des contraintes quadratiques sur les entrées.

Enfin, pour rappel, les opérateurs de mesure sont des opérateurs qui ne sont pas nécessairement des projecteurs de rang 1. Pour qu'ils soient de rang 1, il faudrait entre autres que $\tr(\Pi_k) = 1$. On peut considérer qu'on restreint le problème à un cas de rang 1, et dans ce cas rajouter la contrainte que la somme des éléments diagonaux des opérateurs de mesure doit sommer à 1. Cela permet soit de retirer une variable par opérateur de mesure au problème, en l'exprimant par $x_{n+1} = 1 - \displaystyle \sum_{i=1}^{n} x_i$ avec les $x_i$ éléments diagonaux de l'opérateur de mesure, ce qui nécessite une reformulation du problème, soit l'ajout de la contrainte.

\medbreak

% \section{Résolution avec \texttt{ibex}}

% Pour la résolution de ce problème, utilisons la librairie \texttt{ibex} permettant de faire du calcul par intervale, et possède entre autres un outil d'optimisation, \texttt{ibexopt}. Le problème est formulé avec un langage dédié, \texttt{minibex}. Nous avons eu besoin de définir un opérateur additionnel à ceux présents, l'opérateur \texttt{xlog} permettant d'effectuer l'opération $x \times \log(x)$ en redéfinissant $0 \times \log(0) = 0$ pour que les intervalles ne tombent pas à l'infiniquand ils contiennent 0. De plus, \texttt{minibex} ne considère que des problèmes de minimisation, on ré-écrit le problème en prenant la fonction coût opposée : $\max f(x) \Leftrightarrow \min -f(x)$.
% \medbreak
% Le premier test effectué est sur le cas de deux états d'entrée $\ket{\psi_1} = \ket{0}$ et $\ket{\psi_2} = \ket{1}$ ayant pour probabilité respective $p_1 = 0.1$ et $p_2 = 0.9$. Le résultat théorique est connu: les états étant orthogonaux, on doit obtenir les opérateurs de mesure égaux aux opérateurs densité d'entrée. On obtient bien avec \texttt{ibex} les opérateurs suivant:

% $\Pi_1 = \begin{pmatrix} 0 & 0 \\ 0 & 1\end{pmatrix} , \quad \Pi_2 = \begin{pmatrix} 1 & 0 \\ 0 & 0\end{pmatrix},$

% qui correspondent bien à deux opérateurs de mesure orthogonaux. Dans ce cas, l'information mutuelle est comprise dans l'intervalle $I(\rho, \Pi) \in [0.3250, 0.3254]$, avec un temps de calcul de 8.6 millisecondes.
% \medbreak
% Le deuxième cas qu'on peut présenter est le suivant: $\ket{\psi_1} = \ket{0}$ et $\ket{\psi_1} = \ket{+}$ avec comme probabilité respectives $p_1 = 0.1$ et $p_2 = 0.9$. Le résultat théorique n'est pas donné, et on obtient avec \texttt{ibex} le résultat suivant:

% $\Pi_1 = \begin{pmatrix} 0.445 & 0.497 \\ 0.497 & 0.555\end{pmatrix} , \quad \Pi_2 = \begin{pmatrix} 0.555 & -0.497 \\ -0.497 & 0.445\end{pmatrix},$

% avec une information mutuelle comprise dans l'intervalle $I(\rho, \Pi) \in [0.1348, 0.1349]$, et un temps de calcul de 0.79 secondes.

% % On voit bien avec ces deux exemples l'augmentation radicale du temps de calcul quand on passe d'un problème simple, même orthogonal, à un problème plus compliqué.

% En revanche, ibex ralentit fortement dès qu'on sort des cas simples présentés si-dessus, et notament quand on retire la restriction des opérateurs de mesure à des opérateurs densité purs (de trace unitaire). Deux éléments importants sont à l'origine du problème. Tout d'abord, en analysant l'utilisation des ressources cpu lors de la résolution d'un problème, on voit qu'un seul thread est utilisé par \texttt{ibexopt}. Les algorithmes d'optimisation peuvent être parallélisés, et dans le cas de processeurs à plusieurs c\oe urs on pourrait avoir un gain de performance conséquent. Le deuxième élément est en lien avec la convexité de la fonction coût évoqué précédement. On eut alors se limiter aux extremités de la fonction coût pour réduire le nombre de calculs à effectuer. Il faudrait alors implémenter la condition $0 \in [\texttt{grad}]([f])$, ce qui n'est pas prévu de base dans \texttt{ibexopt}.


\pagebreak
\section{Exemple concret}

\subsection{Données du problème}

Considérons deux états purs quantiques:

\begin{align}
    \ket{\psi_1} = \begin{pmatrix}\frac{1}{3} \\{\vrule height 1mm depth 1mm width 0mm}\\ \frac{2 \sqrt{2}}{3}\end{pmatrix} , \quad \ket{\psi_2} = \begin{pmatrix}\frac{1}{\sqrt{2}} \\ {\vrule height 1mm depth 1mm width 0mm} \\ \frac{1}{\sqrt{2}}\end{pmatrix},
\end{align}
avec les probabilités préalables:
\begin{align}
    p_1 &= P(\ket{\psi_1}) = 0.1, \\
    p_2 &= P(\ket{\psi_2}) = 0.9.
\end{align}

Ces deux états peuvent être réécrits sous la forme d'opérateurs densité:

\begin{align}
    \rho_1 = \begin{pmatrix}\frac{1}{9} && \frac{2 \sqrt{2}}{9} \\ {\vrule height 1mm depth 1mm width 0mm} \\ \frac{2 \sqrt{2}}{9} && \frac{8}{9} \end{pmatrix}, \quad \rho_2 = \begin{pmatrix}\frac{1}{2} && \frac{1}{2} \\ {\vrule height 1mm depth 1mm width 0mm} \\ \frac{1}{2} && \frac{1}{2} \end{pmatrix}
\end{align}

% $\rho_1 = \begin{pmatrix}\frac{1}{9} && \frac{2 \sqrt{2}}{9} \\ \frac{2 \sqrt{2}}{9} && \frac{8}{9} \end{pmatrix}$ et $\rho_2 = \begin{pmatrix}\frac{1}{2} && \frac{1}{2} \\ \frac{1}{2} && \frac{1}{2} \end{pmatrix}$.

On a aussi deux opérateurs de mesure inconnus en dimension 2, soit 8 variables inconnues:
\begin{align}
    \Pi_1 &= \begin{pmatrix}a_1 && b_1 + ic_1 \\ b_1 - ic_1 && d_1\end{pmatrix}, \quad \Pi_2 = \begin{pmatrix}a_2 && b_2 + ic_2 \\ b_2 - ic_2 && d_2\end{pmatrix}
\end{align}

On peut déjà calculer l'information mutuelle maximum qui est $H(X)$ \cite{Cover91} puisqu'on a la distribution de probabilités de $X$: 

\begin{align}
    c_{max} = \displaystyle -\sum_j p(X_j) \log_2(p(X_j)) = -0.1\log_2(0.1) - 0.9 \log_2(0.9) = 0.469 \text{ Shannon},
\end{align}

Quelque soient les mesures $\Pi_k$ choisies, on ne pourra pas obtenir une information mutuelle supérieure à $0.469$ qui est l'information maximale en entrée.

On veut résoudre le problème de la maximisation de l'information mutuelle en fonction des $\{\rho\}$ fixés et des $\{\Pi\}$ à ajuster : 

\begin{align}
    &\max\limits_{\Pi_1, \Pi_2} I(\rho_1, \rho_2, \Pi_1, \Pi_2) \\
    % &= -(\tr(\rho_1 . \Pi_1) + \tr(\rho_1 . \Pi_2))\log(\tr(\rho_1 . \Pi_1) + \tr(\rho_1 . \Pi_2)) \\
    % &- (\tr(\rho_2 . \Pi_1) + \tr(\rho_2 . \Pi_2))\log(\tr(\rho_2 . \Pi_1) + \tr(\rho_2 . \Pi_2)) \\
    % &-(\tr(\rho_1 . \Pi_1) + \tr(\rho_2 . \Pi_1))\log(\tr(\rho_1 . \Pi_1) + \tr(\rho_2 . \Pi_1)) \\
    % &- (\tr(\rho_1 . \Pi_2) + \tr(\rho_2 . \Pi_2))\log(\tr(\rho_1 . \Pi_2) + \tr(\rho_2 . \Pi_2))
% 
\Leftrightarrow \max\limits_{\Pi_1, \Pi_2} \big(&-(\alpha_{11} + \alpha_{12})\log(\alpha_{11} + \alpha_{12}) - (\alpha_{21} + \alpha_{22})\log(\alpha_{21} + \alpha_{22}) \nonumber \\
        &- (\alpha_{11} + \alpha_{21})\log(\alpha_{11} + \alpha_{21}) - (\alpha_{12} + \alpha_{22})\log(\alpha_{12} + \alpha_{22}) \nonumber \\
        &+ (\alpha_{11})\log(\alpha_{11}) + (\alpha_{12})\log(\alpha_{12}) + (\alpha_{21})\log(\alpha_{21}) + (\alpha_{22})\log(\alpha_{22}) \big),\nonumber\\
        & \text{avec } \alpha_{jk} = p_j\tr(\rho_j  \Pi_k) ,\nonumber
\end{align}

tel que :

\begin{align}
    \Pi_1 \succeq 0, \Pi_2 \succeq 0,\\
    \Pi_1 + \Pi_2 = I.
\end{align}

On peut réduire considérablement le nombre de variables. En effet, l'équation \ref{eq:contrainte_somme_id} nous indique que les $\{\Pi\}$ se somment à l'identité, on peut donc réécrire $\Pi_2$ en fonction uniquement de $\Pi_1$ pour réduire à 4 variables $\{a_1, b_1, c_1, d_1\}$:

\begin{align}
    \Pi_2 &= I_2 - \Pi_1 = \begin{pmatrix}1 - a_1 && -b_1 - ic_1 \\ -b_1 + ic_1 && 1 - d_1\end{pmatrix}.
\end{align}

Enfin, on voit que dans ce cas particulier les états quantiques d'entrée sont purs et ne comportent aucun terme complexe, ce qui permet immédiatement de retirer le terme $c$ du système puisqu'il ne sera jamais pris en compte. On se retrouve donc au final avec 3 variables $\{a_1, b_1, d_1\}$.

On pose ensuite les contraintes sur ces variables. Tout d'abord, ces variables sont définies sur ces bornes spécifiques: $a_1 \in [0, 0.5]$, $b_1 \in [-1, 1]$, $d_1 \in [0, 1]$. La détermination de la semi-définie positivité passe par les diagonales et le déterminant strictement positifs, d'une part avec les bornes précédentes et d'autre part avec deux nouvelles contraintes sur les 3 variables.
Le problème s'écrit donc :

\begin{align}
    \max\limits_{a_1, b_1, d_1} I(a_1, b_1, d_1) , \nonumber
\end{align}
tel que :
\begin{align}
    &a_1 \in [0, 0.5] , b_1 \in [-1, 1] , d_1 \in [0, 1] , \nonumber \\
    &a_1 \times d_1 - b_1^2 \geq 0 , \nonumber \\
    &(1-a_1) \times (1-d_1) - b_1^2 \geq 0. \nonumber
\end{align}

\subsection{Résolution avec \texttt{ibex}}
On peut tout d'abord résoudre le problème avec \texttt{ibexopt}. C'est un optimiseur garanti sous contraintes, faisant partie de la librairie de calcul par intervalles \texttt{ibex}. Les codes pour la résolution de ce problème sont disponibles en annexe \ref{appendix:ibexopt}. On obtient une information mutuelle de 0.0681 Shannon, avec une précision relative de $1\times10^{-3}$. On obtient les opérateurs de mesure suivant:

$\Pi_1 = \begin{pmatrix} 0.454 & -0.498 \\ -0.498 & 0.546 \end{pmatrix}$, \quad $\Pi_2 = \begin{pmatrix}0.546 & 0.498 \\ 0.498 & 0.454 \end{pmatrix}$,

avec un temps de calcul de $87$ secondes.

Pour la résolution avec \texttt{ibex}, plusieurs modifications et ajouts doivent être faits. Tout d'abord, le langage utilisé par l'outil d'optimisation, \texttt{minibex}, ne considère que les problèmes de minimisation. Cela nécessite simplement de prendre la fonction opposée puisque $\max(f(x)) \Leftrightarrow \min(-f(x))$. On doit ensuite prendre l'opposé de l'intervalle solution pour obtenir le résultat final. Ensuite, le calcul par intervalle fait apparaître une difficulté quant à l'information mutuelle. On se retrouve à avoir de façon répétée des termes du type $x\times\log(x)$. Cette fonction est clairement définie sur $[0, +\infty[$ mais ne l'est pas sur $]-\infty, 0[$. De part le pessimisme du calcul par intervalle, on peut se retrouver à avoir des intervalles complètement négatifs ou contenant $0$ en entrée de la fonction. On se retrouve donc avec un comportement mal défini dans certains cas. La solution pour ibex est de réimplémenter l'opérateur \texttt{xlog} en considérant qu'en dehors ou sur le bord des bornes de définition on considère que la fonction renvoie $0$. 

Un autre point à noter est la grande variabilité des temps de calculs suivant les problèmes qui sont traités.

\subsection{Résolution avec notre optimiseur}
On a rencontré plusieurs problèmes avec ibex qui nous ont poussé à développer notre propre optimiseur. 

Tout d'abord, en étudiant la fonction information mutuelle, on voit que cette fonction est convexe. Cela permet de considérablement réduire l'espace de définition en cherchant le maximum sur le bord de la fonction au lieu de chercher sur toute la fonction. Ceci n'est pas implémenté par défaut sur ibex.

Ensuite, sur l'implémentation de l'algorithme d'optimisation, on peut voir qu'\texttt{ibexopt} n'utilise qu'un seul thread pour son travail. On l'a vu en figure \ref{fig:algomaxim}, la première boucle est facilement parallélisable, ce qui permet d'accélérer considérablement le temps de résolution, et ce en fonction du matériel à disposition.


Enfin, \texttt{ibexopt} ne permet pas l'accès aux boites en place au cours de l'évolution de l'algorithme. Cela ne permet pas une visualisation plus détaillée de ce qui est effectué, quelles régions sont enlevées au fur et à mesure. Dans le cas de problèmes "simples", ce n'est pas forcément important, en revanche avec un problème comme l'information mutuelle, où le comportement en fonction des variables d'entrée n'est pas immédiatement visible, c'est utile pour pouvoir avoir une idée des influences des contraintes. Un exemple de cette visualisation est disponible en figure \ref{fig:visu_dotnet}. Les coordonnées $(x, y, z)$ correspondent aux variables $(a_1, b_1, d_1)$, et les boites sont colorées en fonction des contraintes satisfaites.

Notre optimiseur nous donne les mêmes solutions qu'ibex, à la différence qu'on arrive à une plus grande précision bien plus rapidement, en $13$ secondes pour les mêmes données.

On peut noter entre autres, à la fois sur \texttt{ibexopt} et sur notre optimiseur que les temps de calculs dépendent du problème qu'on a. La figure \ref{fig:temps_optim} montre par exemple que, plus l'angle entre les deux états s'approche de $90$ degrés et des $180$ degrés, qui sont des cas immédiats de problèmes orthogonaux, plus les temps de résolution sont bien moindres.

\begin{figure}[H]
    \centering
    \includegraphics[scale=0.3]{pb/visu_dotnet.png}
    \caption{Interface web de visualisation de l'optimisation}
    \label{fig:visu_dotnet}
\end{figure}

\begin{figure}[H]
    \centering
    \includegraphics[scale=0.27]{pb/temps_optim.png}
    \caption{Temps d'optimisation en fonction de l'angle entre $\rho_1$ et $\rho_2$}
    \label{fig:temps_optim}
\end{figure}
% \input{includes/chap-05-perpectives.tex}
% \chapter{Ouverture vers le stage}

A la suite de ce travail, un certain nombre de questions restent ouvertes et servirons de pistes de travail pour le stage qui suit:

\begin{itemize}
    \item Sur la construction effective des circuits, est-il optimal de décomposer le circuit en portes élémentaires CNOT ou est-il plus rapide de juste exécuter le problème classiquement ? La question se pose en particulier pour l'algorithme de Deutsch-Jozsa où l'oracle a besoin, a priori, d'être construit en connaissant déjà toutes les sorties de la fonction.
    \item Sur l'algorithme de Deutsch-Jozsa encore une fois, que se passe-t-il quand on l'applique à d'autres classes de fonction, et comment peut-on l'adapter pour différencier d'autres types de fonctions ?
    \item Sur le problème de Deutsch-Jozsa, peut-on trouver un autre algorithme permettant de résoudre le problème ? Cela amène vers l'écriture des contraintes algébriques necessaires à la résolution du problème, et d'essayer de déterminer si la solution proposée par Deutsch-Jozsa est optimale ou non.
    \item En étendant le point précédent, peut-on poser des contraintes algébriques sur les problèmes ou écrire des spécifications algébriques pour les différents algorithmes ? Cela permettrait de caractériser systématiquement les problèmes pouvant disposer d'une amélioration quantique.
\end{itemize}
\chapter{Conclusion}

% Perspectives:
% Etendre le problème à des entrées \{\rho_j\} inconnues => problème bruité. Pose problème de la mauvaise montée en dimension de l'algorithme, donc necessite probablement d'autres améliorations pour permettre résolution en temps correct.
% Utiliser H(X) - H(X|Y). Necessite de coder la fonction H(X|Y) qui est de la forme x*log(y) avec x != y en général. Fonction 2D plus compliquée à formaliser en pratique. Mais réduirait le nombre de termes de la fonction donc possiblement le pessimisme. En revanche, cette fonction a des éléments complexes.

On a ainsi pu voir comment répondre au problème de la détection optimale quantique en utilisant un critère différent de ceux utilisés, l'information mutuelle. Une approche par intervalle, nouvelle pour cette question, a été appliquée pour obtenir un encadrement garanti d'une part du maximum atteint par le critère et d'autre part de la mesure optimale.

\medbreak

Deux implémentations sont proposées, la première en utilisant un optimiseur existant, \texttt{ibexopt}, nécessitant certaines modifications pour s'adapter au problème posé. Un deuxième est proposé, en C\#, améliorant entre autres la vitesse de résolution par rapport à \texttt{ibexopt}.

\medbreak

Les implémentations proposées ne traitent que du cas avec deux états d'entrée et deux opérateurs de mesure, le tout sans termes complexe, et la résolution n'est déjà pas immédiate. En testant sur déjà trois états d'entrée et trois opérateurs de mesure (code \texttt{ibexopt} en annexe \ref{appendix:ibex_3}), les temps de résolution deviennent bien trop important. Il y a donc du travail à faire de ce côté si on veut augmenter en nombre d'états à détecter ou nombre d'opérateurs de mesure à construire. 

% Par exemple, il serait intéressant de coder la fonction $x \times \log(y)$ pour pouvoir utiliser la forme $H(X) - H(X | Y)$ de l'information mutuelle, qui pourrait présenter moins de redondance des variables d'entrées au travers des logarithmes et donc réduire le pessimisme de l'évaluation. Cette fonction est en 2 dimensions, donc moins évidente à représenter correctement par intervalles que la fonction simple $x \times \log(x)$.

\medbreak

Ce problème de détection optimale n'a en revanche été traité que dans son cas le plus simple, en considérant un système "parfait" où aucun bruit n'apparaît. Il serait intéressant de voir plus loin en considérant les états d'entrée bruités. Le problème serait alors de trouver la meilleure configuration états d'entrée - opérateurs de mesure pour que, en présence de bruit, on obtienne une communication optimale.

\begin{appendix}
\chapter{Dynamique des systèmes quantiques}
\label{appendix:dynamic}

Comme n'importe quel système physique, on peut faire évoluer un système quantique dans le temps. L'évolution d'un système quantique est effectuée via une évolution linéaire de son vecteur d'état. Cette évolution linéaire est représentée par un opérateur linéaire sur $\mathcal{H}$, donc par une matrice a partir du moment où une base de référence a été choisie. Cette évolution linéaire doit également rester en accord avec le premier principe \ref{postulat:1}, c'est-à-dire conserver la norme unité du vecteur d'état. La matrice d'évolution doit donc également être unitaire.

% Tout d'abord, il découle du premier postulat \ref{postulat:1} que la dynamique d'un système quantique doit conserver la norme unité. En effet, un état quantique doit, pour être valide, avoir une norme de 1, et donc l'évolution d'un système quantique d'un premier état vers un autre doit conserver cette unitarité. Cela veut dire que la matrice représentant l'évolution du système quantique doit respecter la propriété suivante:

% \begin{equation}
%     U^{\dagger}U = UU^{\dagger} = I,
% \end{equation}

% avec $U$ la matrice d'évolution du système, $U^{\dagger}$ la matrice conjuguée transposée, ou adjointe, de $U$, et $I$ l'identité.

% Une deuxième propriété est également posée, ne découlant pas des deux postulats précédents. La dynamique d'un système quantique doit être aussi linéaire. Ainsi, on pourrait penser que n'importe quelle évolution unitaire serait valable, mais l'expérience nous montre que non, il faut en plus qu'elle soit linéaire.

\medbreak

En pratique, ces évolutions de systèmes quantiques peuvent être réalisées par des portes logiques de façon similaire à la logique booléenne classique. Ces portes quantiques sont complètement caractérisées par la façon dont elles transforment les états quantiques dans la base canonique. On peut alors utiliser des tables de vérité pour les définir, de la même façon qu'en informatique classique:

\begin{enumerate}
    \item La porte de Hadamard $H$. Elle permet de passer un qubit d'un état de base $\ket{0}$ à l'état superposé $\frac{1}{\sqrt{2}}\ket{0} + \frac{1}{\sqrt{2}}\ket{1}$, ou de l'état de base $\ket{1}$ à l'état superposé $\frac{1}{\sqrt{2}}\ket{0} - \frac{1}{\sqrt{2}}\ket{1}$. Elle est très utilisée en début de circuit pour préparer les qubits entrants dans un état permettant l'évaluation parallèle de toutes les entrées;
    \item Les portes de Pauli $X$, $Y$ et $Z$ permettant d'effectuer des rotations aux états des qubits;
    \item La porte de Toffoli, similaire d'un \texttt{NON} booléen à 3 qubit (il effectue un \texttt{NON} sur le dernier qubit quand les deux premiers sont à $\ket{1}$), est une porte universelle quantique \cite{shi2002toffoli}. Elle permet donc de construire l'ensemble des autres portes faisables.
\end{enumerate}

Avec ces portes, on vient construire des circuits quantiques permettant de réaliser des algorithmes. L'annexe \ref{appendix:ConstructionCircuit} montre un exemple de technique de réalisation de circuits. En algorithmes majeurs, on peut citer:

\begin{enumerate}
    \item L'algorithme de Deutsch-Jozsa \cite{Deutsch92} qui permet de résoudre en une opération le problème de différentiation entre une fonction booléenne constante et une fonction booléenne équilibrée. Il faut classiquement $2^n -1$ opérations pour résoudre ce problème.
    \item L'algorithme de Grover \cite{Grover96} qui permet de résoudre en $\mathcal{O}(\sqrt{N})$ opérations le problème de recherche dans une liste non triée. Il faut classiquement au pire $N$ opérations pour effectuer une recherche dans une liste non triée.
    \item L'algorithme de Shor \cite{Shor97} qui permet de résoudre le problème de factorisation en nombres premiers. C'est un problème classiquement très difficile à résoudre, de complexité exponentielle.
\end{enumerate}

Ces trois algorithmes montrent les gains de performance que permet d'obtenir le calcul quantique, qui sont inacessibles avec les technologies d'informatique classique.
\input{includes/app-01-constrcircuit.tex}
\chapter{Codes développés pour \texttt{ibexopt}}
\label{appendix:ibexopt}

\section{Optimisation avec deux états d'entrée}
\begin{lstlisting}[style=CStyle]
Constants
    m_size = 2;
    P1[m_size][m_size] = ((0.0111111111, 0.0314269681); (0, 0.0888888889));
    P2[m_size][m_size] = ((0.45, 0.45); (0, 0.45));

// M: measurement operator
// P: density operator
function tr(M[m_size][m_size], P[m_size][m_size])
    return P(1)(1) * M(1)(1) + 2*M(1)(2)*P(1)(2) + 2*M(2)(1)*P(2)(1) + M(2)(2)*P(2)(2);
end

function EntropyM(_M1[m_size][m_size], _M2[m_size][m_size], _P1[m_size][m_size], _P2[m_size][m_size])
    sum_m1 = tr(_M1, _P1) + tr(_M1, _P2);
    sum_m2 = tr(_M2, _P1) + tr(_M2, _P2);
    return -xlog(sum_m1)-xlog(sum_m2);
end

function EntropyP(_M1[m_size][m_size], _M2[m_size][m_size], _P1[m_size][m_size], _P2[m_size][m_size])
    sum_p1 = tr(_M1, _P1) + tr(_M2, _P1);
    sum_p2 = tr(_M1, _P2) + tr(_M2, _P2);
    return -xlog(sum_p1)-xlog(sum_p2);
end

function EntropyMP(_M1[m_size][m_size], _M2[m_size][m_size], _P1[m_size][m_size], _P2[m_size][m_size])
    return -xlog(tr(_M1, _P1))-xlog(tr(_M1, _P2))-xlog(tr(_M2, _P1))-xlog(tr(_M2, _P2));
end

function MutualInformation(_M1[m_size][m_size], _M2[m_size][m_size], _P1[m_size][m_size], _P2[m_size][m_size])
    return EntropyM(_M1, _M2, _P1, _P2) + ( -0.10 * ln(0.10) -0.90 * ln(0.90)) - EntropyMP(_M1, _M2, _P1, _P2);
end

function I(_P1[m_size][m_size], _P2[m_size][m_size], _m1a, _m1b, _m1c, _m1d)
    _M1 = ((_m1a, _m1b); (_m1c, _m1d));
    _M2 = ((1-_m1a, -_m1b); (-_m1c, 1-_m1d));
    return MutualInformation(_M1, _M2, _P1, _P2);
end

Variables
    M1_a in [0, 0.5],  M1_b in [-1, 1], M1_d in [0, 1];

Minimize
    -I(P1, P2, M1_a,  M1_b,  0, M1_d)

Constraints
    M1_a >= 0;
    M1_a <= 0.5;
    M1_d >= 0;
    M1_a*M1_d - M1_b^2 >= 0;
    (1-M1_a)*(1-M1_d) - (-M1_b)^2 >= 0;
end
\end{lstlisting}

\section{Optimisation avec trois états d'entrée}
\label{appendix:ibex_3}

\begin{lstlisting}[style=CStyle]
Constants
    m_amount = 2;
    m_size = 2;
    P1[m_size][m_size] = ((0.10, 0); (0, 0));
    P2[m_size][m_size] = ((0.3, 0.3); (0, 0.3));
    P3[m_size][m_size] = ((0, 0); (0, 0.3));
  
// M: measurement operator
// P: density operator
function tr(M[m_size][m_size], P[m_size][m_size])
    return P(1)(1) * M(1)(1) + 2*M(1)(2)*P(1)(2) + 2*M(2)(1)*P(2)(1) + M(2)(2)*P(2)(2);
end

function EntropyM(_M1[m_size][m_size], _M2[m_size][m_size], _M3[m_size][m_size], _P1[m_size][m_size], _P2[m_size][m_size], _P3[m_size][m_size])
    sum_m1 = tr(_M1, _P1) + tr(_M1, _P2) + tr(_M1, _P3);
    sum_m2 = tr(_M2, _P1) + tr(_M2, _P2) + tr(_M2, _P3);
    sum_m3 = tr(_M3, _P1) + tr(_M3, _P2) + tr(_M3, _P3);
    return -xlog(sum_m1)-xlog(sum_m2) - xlog(sum_m3);
end

function EntropyP(_M1[m_size][m_size], _M2[m_size][m_size], _M3[m_size][m_size], _P1[m_size][m_size], _P2[m_size][m_size], _P3[m_size][m_size])
    sum_p1 = tr(_M1, _P1) + tr(_M2, _P1) + tr(_M3, _P1);
    sum_p2 = tr(_M1, _P2) + tr(_M2, _P2) + tr(_M3, _P2);
    sum_p3 = tr(_M1, _P3) + tr(_M2, _P3) + tr(_M3, _P3);
    return -xlog(sum_p1)-xlog(sum_p2)-xlog(sum_p3);
end

function EntropyMP(_M1[m_size][m_size], _M2[m_size][m_size], _M3[m_size][m_size], _P1[m_size][m_size], _P2[m_size][m_size], _P3[m_size][m_size])
    return -xlog(tr(_M1, _P1))-xlog(tr(_M1, _P2))-xlog(tr(_M1, _P3))-xlog(tr(_M2, _P1))-xlog(tr(_M2, _P2))-xlog(tr(_M2, _P3))-xlog(tr(_M3, _P1))-xlog(tr(_M3, _P2))-xlog(tr(_M3, _P3));
end

function MutualInformation(_M1[m_size][m_size], _M2[m_size][m_size], _M3[m_size][m_size], _P1[m_size][m_size], _P2[m_size][m_size], _P3[m_size][m_size])
    return EntropyM(_M1, _M2, _M3, _P1, _P2, _P3) + ( -0.10 * ln(0.10) -0.60 * ln(0.60)-0.30 * ln(0.30)) - EntropyMP(_M1, _M2, _M3, _P1, _P2, _P3);
end

function I(_P1[m_size][m_size], _P2[m_size][m_size], _P3[m_size][m_size], _m1a, _m1b, _m1c, _m1d, _m2a, _m2b, _m2c, _m2d)
    _M1 = ((_m1a, _m1b); (_m1c, _m1d));
    _M2 = ((_m2a, _m2b); (_m2c, _m2d));
    _M3 = ((1-_m1a-_m2a, -_m1b-_m2b); (-_m1c-_m2c, 1-_m1d-_m2d));
    return MutualInformation(_M1, _M2, _M3, _P1, _P2, _P3);
end

Variables
    M1_a in [0, 1],  M1_b in [-1, 1], M1_c in [-1, 1], M1_d in [0, 1];
    M2_a in [0, 1],  M2_b in [-1, 1], M2_c in [-1, 1], M2_d in [0, 1];

Minimize
    -I(P1, P2, P3, M1_a,  M1_b,  M1_c, M1_d, M2_a,  M2_b,  M2_c, M2_d);

Constraints
    M1_a*M1_d - M1_b^2 - M1_c^2 >= 0;
    M1_a <= 1/3; // cost function & constraint both symmetric ((M1, M2) <=> (M2, M1))
    M2_a <= 0.5;
    M1_a + M1_d = 1;
    M2_a + M2_d = 1;
    M2_a*M2_d - M2_b^2 - M2_c^2 >= 0;
    (1-M1_a-M2_a) >= 0;
    (1-M1_a-M2_a) <= 1;
    (1-M1_d-M2_d) >= 0;
    (1-M1_d-M2_d) <= 1;
    (1-M1_a-M2_a)*(1-M1_d-M2_d) - (-M1_b-M2_b)^2 - (-M1_c-M2_c)^2 >= 0;
end
\end{lstlisting}    
\end{appendix}


\clearpage
\addcontentsline{toc}{chapter}{Bibliographie}
\bibliographystyle{ieeetr}
\bibliography{biblio}

\clearpage
\thispagestyle{empty}

\vspace*{\fill}
\noindent\rule[2pt]{\textwidth}{0.5pt}\\
{\textbf{Résumé ---}}
Ce mémoire s'intéresse à la question de la détection optimale quantique. On dispose d'un ensemble d'états quantiques que l'on veut détecter le plus efficacement possible par le biais d'un ensemble d'opérateurs de mesure. Plusieurs critères de performance existent tels que l'erreur de mesure ou l'erreur quadratique. On étudie ici le critère de l'information mutuelle, en y appliquant les techniques de calcul par intervalles pour obtenir un résultat global garanti.

{\textbf{Mots clés :}}
Informatique quantique, détection optimale, opérateur de mesure, information mutuelle, calcul par intervalles.
\\
\noindent\rule[2pt]{\textwidth}{0.5pt}


\vspace*{\fill}
\noindent\rule[2pt]{\textwidth}{0.5pt}\\
{\textbf{Abstract ---}}
This report presents the problem of optimal quantum detection. We are provided with a set of quantum states with prior probabilities, and we want to build the set of quantum measurement operators to have an optimal measure. Multiple performance criterion exist, such as the probability of error of detection, or the square root error. We use the mutual information, using interval analysis to provide a global guaranteed solution. 

{\textbf{Keywords :}}
Quantum computing, quantum optimal detection, measurement operator, mutual information, interval analysis.
\\
\noindent\rule[2pt]{\textwidth}{0.5pt}

\begin{center}
  Polytech Angers\\
  62, avenue Notre Dame du Lac\\
  49000 Angers
\end{center}
\vspace*{\fill}

\end{document}