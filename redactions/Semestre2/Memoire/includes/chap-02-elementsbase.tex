\chapter{Informatique quantique: éléments de base}


Les notions de base d'informatique quantique sont décrites dans plusieurs ouvrages de référence, notamment dans \cite{Nielsen00, Mermin07}. On présente ici un résumé des notions fondamentales à connaître pour la suite du rapport.

\section{Postulats de base}

On pose 3 postulats, servant de base aux raisonnements qui suivront. Ces postulats sont confirmés jusqu'à présent par les expériences.

\subsection{Etat d'un système quantique}
Un système quantique peut être représenté par un vecteur d'état, de la même manière qu'un système physique classique. On le représente par la notation de Dirac, notée de la forme $\ket{\psi}$. Ce vecteur d'état est nécessairement de norme 1 (la somme des modules au carré vaut 1). On peut distinguer deux types d'états pour un système quantique: les états de base, formant une base orthonormée d'un espace vectoriel complexe, et les états superposés. Ces états superposés correspondent à une combinaison linéaire des états de base. On peut écrire généralement un état quantique de la façon suivante:

\begin{equation}
    \ket{\psi} = \displaystyle\sum_{i} c_i \ket{k_i},
\end{equation}

avec les $\ket{k_i}$ états de base, et les $c_i$ respectant $ \displaystyle\sum_{i} |c_i|^2 = 1$ pour la normalisation du vecteur d'état.

\medbreak

Dans le cadre de l'informatique quantique, on utilise le système quantique le plus simple, appelé \textbf{qubit}. Ce système quantique est composé de deux états de base, $\ket{0}$ et $\ket{1}$, et des états superposés. Similairement à l'informatique classique, où on travaille sur le système physique le plus élémentaire - le bit - en quantique on travaille sur le système physique quantique élémentaire - le qubit. On dispose des mêmes états de base, mais l'informatique quantique apporte les états \textit{intermédiaires} superposés. Dans la base canonique $\{\ket{0}, \ket{1}\}$, on note un qubit de la façon suivante: $\ket{\psi} = \alpha \cdot \ket{0} + \beta \cdot \ket{1}$.

\subsection{Mesure projective}
Que se passe-t-il quand on mesure un système quantique ? On a évoqué au dessus la notion de superposition des états. L'expérience montre que, lorsqu'on va mesurer un système quantique, on va mesurer au hasard un des états de base, avec comme probabilité le carré du coefficient correspondant.

Mathématiquement, la mesure effectue une projection de l'état du système sur un des états de base dont il est composé. Par exemple, si on a un qubit dans l'état $\ket{\psi} = \frac{1}{\sqrt{2}}\ket{0} + \frac{1}{\sqrt{2}}\ket{1}$, alors la probabilité de mesurer 0, c'est-à-dire de projeter le système dans l'état de base $\ket{0}$ est $|\frac{1}{\sqrt{2}}|^2 = \frac{1}{2}$; et de même pour l'état de base $\ket{1}$. On a donc exactement la même probabilité de mesurer $\ket{0}$ que de mesurer $\ket{1}$.

Il faut noter que, lorsqu'on fait la mesure, on projette réellement le système quantique dans l'état de base. Concrètement, si on a un état superposé qu'on mesure, il se place dans l'état de base qu'on mesure, et toutes les mesures successives qu'on fera sur ce qubit donneront le même résultat. La mesure fait donc perdre l'état qu'on avait auparavant.

\subsection{Dynamique du système}
Comme n'importe quel système physique, on peut faire évoluer un système quantique dans le temps. Ici apparaissent deux propriétés. Tout d'abord, il découle du premier postulat que la dynamique d'un système quantique doit conserver la norme unité. En effet, un état quantique doit, pour être valide, avoir une norme de 1, et donc l'évolution d'un système quantique d'un premier état vers un autre doit conserver cette unitarité. Cela veut dire que la matrice représentant l'évolution du système quantique doit respecter la propriété suivante:

\begin{equation}
    U^{\dagger}U = UU^{\dagger} = I,
\end{equation}

avec $U$ la matrice d'évolution du système, $U^{\dagger}$ la matrice conjuguée transposée, ou adjointe, de $U$, et $I$ l'identité.

Une deuxième propriété est également posée, ne découlant pas des deux postulats précédents. La dynamique d'un système quantique doit être aussi linéaire. Ainsi, on pourrait penser que n'importe quelle évolution unitaire serait valable, mais l'expérience nous montre que non, il faut en plus qu'elle soit linéaire.
