\chapter{Introduction}

Le problème de la détection optimale est un problème classique du traitement de l'information, issu des travaux sur les signaux dans la deuxième moitié du $\text{XX}^{\text{ème}}$ siècle. Le problème est de pouvoir détecter du mieux que l'on puisse un signal contenant de l'information parmi un signal plus ou moins quelconque: Alice envoie à Bob un message via un canal de transmission, et Bob doit pouvoir retrouver optimalement le message transmis par Alice.

On s'intéresse au problème équivalent dans le cadre de l'information quantique: Alice envoie à Bob un message codé sur un support quantique, et Bob doit pouvoir retrouver optimalement le message transmis. 
Alice en entrée choisit d'envoyer un signal quantique sélectionné parmi un ensemble d'états possible avec une distribution de probabilités spécifiée. Bob en sortie met en \oe uvre une mesure quantique du signal reçu. Il s'agit alors de déterminer la mesure quantique optimale qui permet d'identifier le plus efficacement possible les états envoyés.

Ce problème peut se traiter en utilisant un certain nombre de critères de performance. Eldar, notamment, a utilisé le critère de la probabilité de détection correcte à maximiser \cite{Eldar03c}, ce qui revient à un problème de maximisation linéaire sous contraintes semi-définies positives, facile en pratique à résoudre. De même, Eldar a aussi considéré le critère de l'erreur quadratique de mesure \cite{Eldar01}, ce qui revient à un problème de minimisation quadratique sous les mêmes contraintes semi-définies positives.

Nous nous intéressons ici au critère de l'information mutuelle \cite{Davies78}, qui pose un problème de maximisation de fonction non linéaire sous contraintes semi-définies positives. Ce problème n'est en pratique pas facile à résoudre. Il peut l'être de façon approximative par exemple par des solutions de recuit. Nous nous intéressons à la résolution de ce problème de façon garantie, en utilisant le calcul par intervalle pour fournir un encadrement garanti, global, de la solution optimale. Nous allons donc ainsi examiner dans ce travail, de façon originale pour la première fois à notre connaissance, l'application du calcul par intervalles pour la conception de détecteurs quantiques optimaux en présence d'un critère de performance informationnel non linéaire.

Ce rapport détaille dans un premier chapitre les éléments d'information quantique relatifs au problème traité. On détaillera par la suite les notions de calcul par intervalle et l'algorithme d'optimisation utilisé. En dernier lieu, on détaillera le problème d'optimisation des détecteurs quantiques et l'application de l'analyse par intervalles à celui-ci.