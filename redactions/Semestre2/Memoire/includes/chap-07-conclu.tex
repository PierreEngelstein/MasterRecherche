\chapter{Conclusion}

% Perspectives:
% Etendre le problème à des entrées \{\rho_j\} inconnues => problème bruité. Pose problème de la mauvaise montée en dimension de l'algorithme, donc necessite probablement d'autres améliorations pour permettre résolution en temps correct.
% Utiliser H(X) - H(X|Y). Necessite de coder la fonction H(X|Y) qui est de la forme x*log(y) avec x != y en général. Fonction 2D plus compliquée à formaliser en pratique. Mais réduirait le nombre de termes de la fonction donc possiblement le pessimisme. En revanche, cette fonction a des éléments complexes.

On a ainsi pu voir comment répondre au problème de la détection optimale quantique en utilisant un critère différent de ceux utilisés, l'information mutuelle. Une approche par intervalle, nouvelle pour cette question, a été appliquée pour obtenir un encadrement garanti d'une part du maximum atteint par le critère et d'autre part de la mesure optimale.

\medbreak

Deux implémentations sont proposées, la première en utilisant un optimiseur existant, \texttt{ibexopt}, nécessitant certaines modifications pour s'adapter au problème posé. Un deuxième est proposé, en C\#, améliorant entre autres la vitesse de résolution par rapport à \texttt{ibexopt}.

\medbreak

Les implémentations proposées ne traitent que du cas avec deux états d'entrée et deux opérateurs de mesure, le tout sans termes complexe, et la résolution n'est déjà pas immédiate. En testant sur déjà trois états d'entrée et trois opérateurs de mesure (code \texttt{ibexopt} en annexe \ref{appendix:ibex_3}), les temps de résolution deviennent bien trop important. Il y a donc du travail à faire de ce côté si on veut augmenter en nombre d'états à détecter ou nombre d'opérateurs de mesure à construire. 

% Par exemple, il serait intéressant de coder la fonction $x \times \log(y)$ pour pouvoir utiliser la forme $H(X) - H(X | Y)$ de l'information mutuelle, qui pourrait présenter moins de redondance des variables d'entrées au travers des logarithmes et donc réduire le pessimisme de l'évaluation. Cette fonction est en 2 dimensions, donc moins évidente à représenter correctement par intervalles que la fonction simple $x \times \log(x)$.

\medbreak

Ce problème de détection optimale n'a en revanche été traité que dans son cas le plus simple, en considérant un système "parfait" où aucun bruit n'apparaît. Il serait intéressant de voir plus loin en considérant les états d'entrée bruités. Le problème serait alors de trouver la meilleure configuration états d'entrée - opérateurs de mesure pour que, en présence de bruit, on obtienne une communication optimale.